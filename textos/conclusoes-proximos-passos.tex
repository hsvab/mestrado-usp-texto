% ---
% Capitulo Considerações Finais
% ---
\chapter{Considerações Finais}\label{chap:considfinais}

% META: 5p.

Um dos principais indicadores de mobilidade, o total de viagens que uma pessoa realiza num dia, é fortemente influenciado pela renda familiar, pelo grau de instrução do indivíduo e, em menores medidas, pelas presenças de crianças até 14 anos e de dois ou mais automóveis na família.
Essas influências, porém, não se dão de forma homogênea em toda a população, incidindo diferentemente sobre os grupos sociais.
Essas conclusões foram possíveis após análise de um conjunto de modelos de regressões \textit{quasi-poisson} aplicados a diferentes grupos, determinados pela articulação das variáveis sexo e situação familiar, numa primeira aproximação do gênero como categoria de análise.
Está na origem desta pesquisa a limitação do próprio conjunto de dados (secundários) que aborda especificamente o sexo como variável de interesse, mas não o gênero como categoria de análise.
Assim, a partir deste trabalho, recomenda-se estudar outras combinações de variáveis desse mesmo banco de dados para aprimorar a abordagem de gênero na compreensão da mobilidade na RMSP.
Além disso, é possível elaborar pesquisas de caráter qualitativo para esmiuçar melhor como é o processo decisório dentro do núcleo familiar relativo à compra e ao uso do(s) carro(s), ou ainda, relativo aos deslocamentos decorrentes do cuidados com as crianças. Com isso, poder-se-á ir além das constatações deste trabalho em relação ao comportamento de demanda, em busca das motivações desses comportamentos. Será um grande avanço ainda se houver coleta de dados sobre raça/etnia, possibilitando abordagens interseccionais.

A divisão do trabalho de acordo com o gênero implica diferentes padrões de atividades e, assim, diferentes padrões de viagens. 
Analisando os dados expandidos do total de viagens da pessoa, observou-se que, ao considerar inclusive quem não fez viagens, as mulheres sempre tinham médias inferiores aos homens. Ao passo que se considerarmos apenas quem fez viagem, essa situação se mantém para 1977 e 1987, mas em 1997 e 2007, as mulheres móveis, o são mais do que os homens. 
Elas têm conseguido diminuir as desigualdades no mercado de trabalho ao longo do tempo, mas não vêm obtendo o mesmo êxito em relação ao trabalho doméstico. 
Além da obrigatoriedade na realização das viagens motivo trabalho, elas não ficaram desobrigadas daquelas ligadas ao espaço doméstico, como se pode ver nos maiores percentuais femininos de viagens motivo manutenção/compras para todos os anos.

As mulheres vem utilizando mais o transporte coletivo do que os homens (com exceção do período de 1987), sendo o uso do modo ônibus o a mais representativo.
Então, embora tenham uma diversidade maior de atividades a cumprir, ao se deslocar por meio motorizado, elas utilizam mais frequentemente um modo cujas rotas não são flexíveis.
Isso significa que é possível cumprir uma agenda mais complexa na RMSP sem que seja necessário dispor de um automóvel e que, talvez, o modo que confira maior flexibilidade de rota não seja o carro, seja a pé.
As mulheres caminham mais, considerando o conceito de viagem a pé adotado pelas Pesquisas OD, o que provavelmente sub-representa este tipo de viagem.
Entretanto, para estimular o modo a pé em substituição ao uso do carro, é preciso considerar a principal limitação do modo a pé: as grandes distâncias. 
Assim, políticas de transporte precisam necessariamente ser articuladas com o planejamento urbano para melhor distribuir as oportunidades na cidade. 
O acesso às oportunidades (de trabalho, estudo, lazer, compras, saúde) de forma mais equânime no espaço urbano torna possível a utilização de modos não motorizados, mais sustentáveis. 
E dada a existência de oportunidades mais próximas à residência, é preciso também que o ambiente construído seja convidativo a realizar as viagens a pé ou de bicicleta, ou seja, as pessoas de qualquer gênero devem sentir-se seguras e acolhidas pela cidade que as cerca. 
Posto isso, ao abordar gênero e mobilidade, recomenda-se fortemente estudos focados nas viagens a pé, que invariavelmente precisarão de coletas de dados para além das Pesquisas OD, com ênfase na subjetividade e na preocupação com o contexto urbano.

%Logo, tendo em vista a elaboração de políticas de transporte público que o torne mais atrativo para as mulheres, é preciso primeiramente garantir que o transporte público seja acessível. Isto significa tanto não haver barreiras econômicas (tarifárias), já que mulheres têm rendimento médio inferior a homens, quanto haver capilaridade suficiente da rede, para que haja a percepção de que existe transporte público ``à disposição'', assim como se tem com o carro. Em segundo lugar é preciso que o(s) modo(s) seja(m) adequado(s) à atividade que será desempenhada. Para mulheres, se considerarmos as viagens motivo ``manutenção'' da casa, acompanhar crianças à escola ou ao médico e ir às compras são atividades relevantes. No Reino Unido, Hamilton e Jenkins (2000) apontam para a falta de adequação da infraestrutura de transportes às necessidades socialmente atribuídas às mulheres. No Brasil, o diagnóstico não difere à exceção de poucos municípios como Curitiba: é impossível uma mulher utilizar um ônibus com um carrinho de bebê ou carrinho de compras. 

Neste estudo, observou-se que, apesar da variável ano não entrar nas clusterizações, ela foi muito relevante na formação dos grupos, indicando que o efeito do tempo pode ter grande peso nos padrões de deslocamentos.
Pelos dados das quatro Pesquisas OD observou-se que o número total de viagens por família diminuiu enquanto os tempos e distâncias de viagem por pessoa aumentaram.
As distâncias de viagem aumentaram mais para o usuário do transporte coletivo do que para o individual e, especificamente dentro dos modos coletivos, o efeito do alongamento das viagens é mais sentido na alta capacidade (metrô e trem) - provavelmente devido à expansão da rede metroferroviária ocorrida no período.
Quando ocorre, a taxa de incremento das durações e distâncias é maior para viagens de motivação compulsória (trabalho e educação).
As durações de viagem aumentaram tanto para quem usa modos coletivos como para quem usa os individuais, afetando todos motivos.
O quadro de tendência de aumento de durações e distâncias indica que houve expansão da RMSP, mas sobretudo mostra que a capacidade da malha de transportes oferecida não tem acompanhado a demanda.
Cabem, por conseguinte, estudos futuros que aprofundem as análises longitudinais, buscando avaliar os impactos de efeitos fixos e aleatórios e orientar o planejamento de transportes na busca por mais eficiência.

%Dentro no grupo dos modos motorizados, tornar o transporte público mais acessível a todos(as) é uma condição para atingir padrões de deslocamentos mais sustentáveis. Entre as iniciativas possíveis nesse sentido é preciso considerar a questão da promoção da equidade. Nas decisões relativas à infra-estrutura do sistema de transporte quais modos terão prioridade nos espaço de circulação viário, como promover a capilarização da rede e como tornar financeiramente acessível as tarifas a todas pessoas.

Os padrões de viagem se alteram conforme o tempo, o que foi explorado nas análises de \textit{clusters} e com regressões logísticas, e conforme o gênero (situações exploradas com regressões \textit{quasi-poisson}). 
Portanto, o objetivo principal desta dissertação foi atingido ao constatar que a transformação dos papeis sociais desempenhados por homens e mulheres dentro do núcleo familiar e na sociedade, ao longo das últimas décadas, alterou de maneira significativa a maneira como pessoas com identidades de gênero masculina e feminina têm se deslocado.

  
%Por fim, constatando a influência da divisão do trabalho sobre o padrão de mobilidade de acordo com o gênero, podem ser tomadas medidas de incentivo ao uso do transporte público de forma a considerar a diversidade de necessidades que levam as pessoas a se locomoverem nas cidades (GTZ, 2007). São exemplos de medidas que podem ser adotadas: incremento da infra-estrutura do sistema de transporte existente, capilarização da rede, mudança na metodologia de pesquisa origem-destino para que se passe a considerar viagens a pé inferiores a 500m ou trechos a pé na cadeia de viagens. Estas mudanças podem aumentar a eficácia e eficiência do sistema de transporte urbano, já que este será mais atrativo - não apenas para as mulheres - e pode contribuir com a diminuição dos congestionamentos.

%Trabalhos futuros:
%Na conexão entre gênero e mobilidade preciso mudar a agenda da pesquisa na direção em que se sintetizem três dimensões: localidade, abordagens quantitativas e qualitativas, e modos de pensar transversais a gênero e mobilidade. 
%ausência de informação sobre a raça/etnia
%espacialização