% ---
% Capitulo Métodos
% ---
\chapter{Métodos}\label{chap:metodo}

%constructo como sendo uma definição mental de uma ideia de pesquisa, estabelecida com base na teoria subjacente e/ou na experiência e na intuição do pesquisador

Esta pesquisa lida fundamentalmente com dados secundários, principalmente do Metrô-SP através de suas Pesquisas OD e, em menor grau, com dados sócio-econômicos advindos de pesquisas PNAD
\footnote{A PNAD é a Pesquisa Nacional por Amostra de Domicílios, realizada bianualmente pelo IBGE (referência no mês de setembro) com objetivo de investigar características socioeconômicas da população. Fonte:\url{http://www.previdencia.gov.br/arquivos/office/3_081014-105206-595.pdf} Acesso em 20 de novembro de 2014} e censos do IBGE
\footnote{O censo demográfico é uma pesquisa realizada decenalmente pelo IBGE com objetivo de caracterizar sociodemograficamente a população brasileira. O primeiro foi feito em 1872 e o último data de 2010. Fonte:\url{http://cod.ibge.gov.br/234lq} Acesso em 20 de novembro de 2014}.
Por dados secundários entendem-se aqueles que ``já foram coletados para objetivos que não os do problema'' \cite[p.127]{MALHORTA2001}. Trabalhar com dados secundários traz vantagens e desvantagens. Como vantagens tem-se o acesso relativamente fácil, o baixo custo de coleta e a rapidez de obtenção dos dados. Como desvantagem tem-se que o propósito da coleta dos dados difere daquele para que estão sendo utilizados aqui. Isto é, a formulação dos questionários o desenho dos bancos de dados buscam responder a perguntas diferentes das propostas por esta dissertação.
Essa desvantagem não inviabiliza o uso dos dados, mas é uma informação que deve estar em mente ao manipulá-los e analisá-los. \citeauthoronline{MALHORTA2001} (\citeyear{MALHORTA2001}, p.128) indica que os dados secundários podem auxiliar a:

\begin{citacao}
	1. identificar o problema\\
	2. definir melhor o problema\\
	3. desenvolver uma abordagem do problema\\
	4. formular uma concepção de pesquisa adequada\\
	5. responder a certas perguntas de pesquisa e testar algumas hipóteses\\
	6. interpretar dados primários com mais critério
\end{citacao}

Nesta dissertação os bancos de dados e manuais de referência das respectivas Pequisas OD foram requisitados em 05 de maio de 2014 por meio de formulário \emph{online} do e-SIC do governo do Estado de São Paulo e disponibilizados pelo Metrô-SP em mídia digital para retirada em \hl{ver com Orlando data da carta junto ao CD} (no Anexo \ref{chap:esic} está disponível a comunicação com o Metrô-SP para obtenção dos bancos de dados). Os dados provenienetes de PNAD, censos ou outras pesquisas foram obtidos por meio de relatórios públicos, disponibilizados \emph{online} e têm suas fontes indicadas ao longo do texto. Entende-se que os dados advindos das Pesquisas OD auxiliarão na tarefa de identificar se existem diferentes padrões de mobilidade de acordo com o gênero na RMSP, pois contêm informações sobre o sexo e sobre os deslocamentos de indivíduos, com representatividade em suas respectivas zonas e indicados os respectivos fatores de expansão nos bancos de dados. Para investigar os motivos dessas diferenças têm-se como base as hipóteses advindas da revisão da literatura. Os dados socieconômicos podem dar indícios daquelas ligadas à influência dos critérios socioeconômicos sobre as ações dos indivíduos. Dados como idade e ocupação (se trabalha, se estuda, etc.) podem dar indícios para compreender influência do ciclo de vida no comportamento dos indivíduos. Dados como situação na família (pessoa responsável, cônjuge, etc.) e presença de filhos podem auxiliar a compreender a dinâmica familiar e seu reflexo nos padrões de atividades das pessoas. Em tempo, dados como posse de automóveis, motocicletas, bicicletas e modos utilizados nas viagens, podem contribuir para compreender a relação que é estabelecida com o transporte público e o transporte privado, mais especificamente o carro. 

Os dados secundários das Pesquisas OD foram analisados segundo alguns critérios, observados no Quadro \ref{qua:dados-sec}. Os objetivos primários da coleta dos dados são apresentados em na Seção \ref{sec:period-obj}. Em relação à natureza dos dados, para que fossem melhor utilizados nesta dissertação, foram feitas compatibilizações entre as zonas (geográficas) de análise e os diversos bancos de dados, processo mais detalhadamente explicado em na Seção \ref{sec:bd}. Em relação à confiabilidade, os dados foram adquiridos diretamente do Metrô-SP, responsável pela coordenação da coleta e compilação dos dados, e que goza de boa reputação e pioneirismo no Brasil na realização de pesquisas dessa natureza. Em relação à atualidade dos dados, esclarece-se na Seções \ref{sec:period-obj} e \ref{sec:descricao-OD} a periodicidade e datas de referência das OD-1977, OD-1987, OD-1997 e OD-2007. Em relação às especificações e metodologia, na Seção \ref{sec:descricao-OD} são descritos brevemente os métodos de coleta, amostragem, conceitos utilizados, além das decorrentes limitações.

\begin{quadro}[htb]
    \IBGEtab{
        \renewcommand{\arraystretch}{1.5}
        \ABNTEXfontereduzida
        \caption[Critérios para Avaliação de Dados Secundários]{\label{qua:dados-sec}Critérios para Avaliação de Dados Secundários}
	}{%
        \begin{tabular}{|P{5.0cm}|P{10.00cm}|}
           \hline
		       \headerCenterCell{Critérios} & 
		       \headerCenterCell{Questões} \\
		    \hline\hline
		        Objetivo&
		        Por que os dados fortam coletados?\\
		    \hline
		        Natureza&
		        Definição de variáveis chave; unidades de medição; categorias usadas e relações examinadas\\
		    \hline
		        Confiabilidade&
		        Experiência; credibilidade; reputação e integridade da fonte\\
		    \hline
		        Atualidade&
		        Prazo entre coleta e publicação; frequência das atualizações\\
			\hline
		        Especificações e Metodologia&
		        Método de coleta de dados; índice de respostas; qualidade dos dados; técnica de amostragem; tamanho da amostra; criação do questionário e trabalho de campo\\
			\hline
		\end{tabular}
	}{%
		\fonte{Adaptado de \cite[p.129]{MALHORTA2001}}
    }
\end{quadro}

Como análises preliminares dos dados, são apresentadas algumas estatísticas feitas para os bancos de dados das quatro \emph{cross-sections} (1977, 1987, 1997 e 2007). Essas estatísticas podem ser divididas em dois grupos: o de caracaterísticas dos indivíduos e o de características das viagens. O primeiro grupo de análises busca compreender como é essa amostra, em cada ano e diferencialmente entre os anos, olhando variáveis como idade, situação na família, grau de instrução e renda. O segundo grupo de análises busca compreender como essa amostra se comporta em termo de viagens realizadas, em cada ano e diferencialmente entre os anos, olhando para tanto variáveis como duração das viagens e número de viagens realizadas.

Dentro deste segundo grupo de análises foi de interese buscar verificiar se a variável explicativa sexo era relevante para explicar tanto a duração como o número de viagens. Para tanto foram feitas regressões lineares simples e seus resultados são apresentados na Seção \ref{sec:analises-preliminares}. També há interesse em verificar se existe alguma diferença estatisticamente significativa nos padrões de deslocamento entre os sexos, para cada \emph{cross-section}. Isso foi feito feito tomando como hipótese nula que as médias de ambos sexos eram iguais, para as variáveis dependentes analisadas. Essa hipótese foi testada e os resultados também são apresentados na Seção \ref{sec:analises-preliminares}.

Por fim, esta pesquisa utiliza um banco de dados secundário em que a informação relativa a gênero nasce quase que exclusivamente da variável sexo, codificada binariamente - certamente uma limitação; muito embora seja o tipo de pergunta que o(a) entrevistador não faz diretamente, mas marca no papel a partir de julgamento visual. Isso significa que uma pessoa transsexual cuja performatividade seja feminina, provavelmente será identificada como mulher no questionário. As variáveis de análise (dependentes) geralmente feitas nas pesquisas desta temática giram em torno de tempo e distância viajados, modos utilizados, motivo das viagens e encadeamento das viagens \cite{HANSON2010}. Estas variáveis começaram a ser exploradas no presente texto e continuarão a ser desenvolvidas e se nesta abordagem comum não houver elemento inovador por conta da metodologia adotada, que seja útil pela sua existência já que não se localizou trabalho dessa natureza feito para a RMSP, cobrindo o período de tempo aqui analisado. 

%da conversa com a Glaucia
%ANOVA p/ distância e tempo (homens e mulheres)
%chi-quadrado p/ verificar a diferença entre os grupos
%SPSS: Analyze > compare means > one way ANOVA
%ticar: LSD, TUKEY
%Data>split file


