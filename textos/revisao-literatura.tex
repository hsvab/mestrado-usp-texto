% ---
% Capitulo Revisão de Literatura
% ---
\chapter{Revisão de Literatura}\label{chap:revisao-literatura}
% ---
\begin{citacao}
	\begin{flushright}  
\emph{``Todo enunciado - desde a breve réplica (monolexemática) até o romance ou o tratado científico - comporta um começo absoluto e um fim absoluto: antes de seu início, há os enunciados dos outros, depois de seu fim, há os enunciados-respostas dos outros [\ldots]. O locutor termina seu enunciado para passar a palavra ao outro ou para dar lugar à compreensão responsiva ativa do outro. O enunciado não é uma unidade convencional, mas uma unidade real, estritamente delimitada pela alternância dos sujeitos falantes'' (Bakhtin)}
	\end{flushright}
\end{citacao}

Este capítulo tem por objetivo clarificar conceitos considerando a evolução e as interesecções entre as concepções utilizadas, bem como dar um panorama geral de como as questões de gênero, de mobilidade e de sustentabilidade vêm sendo tratadas sob a perspectiva do planejamento de transportes. 
Buscou-se, sempre que possível, apresentar aspectos ligados à realidade brasileira e quiçá paulistana, pois o escopo espacial de análise do presente trabalho é a Região Metropolitana de São Paulo (RMSP) (ver Figura \ref{fig:mapa-rmsp}), área coberta pela Pesquisas Origem e Destino (Pesquisas OD) do Metrô-SP. 

\begin{figure}[htb]%
    \caption{\label{fig:mapa-rmsp}Mapa dos municípios que compõem a região metropolitana em 2014, divididos por sub-regiões}%
    \begin{center}%
        \includegraphics[width=0.9\textwidth]{./imagens/Mapa-RMSP-subregions.png}%
    \end{center}%
    \fonte{Mapa elaborado por Marcos Elias Oliveira Júnior, segundo a Lei 1.139/2011 \cite{LEI1139}. Disponível em: \url{http://pt.wikipedia.org/wiki/Regi\%C3\%A3o_Metropolitana_de_S\%C3\%A3o_Paulo\#mediaviewer/File:Mapa-RMSP-subregions.svg} Acesso em 10 de novembro de 2014}
\end{figure}%

\section{Gênero}
% META: 10p.

Ao nascer, umas das primeiras atividades do ser humano é comunicar-se, o que inclui nominar para si o mundo que o cerca. Esse processo não se dá de maneira solitária, a nominação advém de uma interação social que visa compartilhar signos afim de efetivar a comunicação. Por isso, este capítulo visa estabelecer o que se deseja exprimir através de palavras-chave deste trabalho (gênero, mobilidade e sustentabilidade), considerando a visão de Bakthin, cujo trabalho, segundo \citeauthoronline{STELLA2005} (\citeyear{STELLA2005}), indicava ser necessário

\begin{citacao}
não somente a palavra, mas também a linguagem em geral, ser concebida e tratada de uma outra forma, levando-se em conta sua história, sua historicidade, ou seja, especialmente a linguagem em uso. Isso significa que, no pensamento bakthiniano, a palavra reposiciona-se em relação às concepções tradicionais, passando a ser encarada como um elemento concreto de feitura lógica.
\end{citacao}

Há um senso comum que confunde e funde, não por acaso, os conceitos de sexo e gênero, muito embora sejam distintos - distinção esta encontrada em maior ou menor grau de acordo com o idioma. Em inglês a palavra \emph{sex} tem sentido mais limitado, ligado à anatomia, e a palavra \emph{gender} tem sentido mais amplo, ligado à construção cultural da identidade. Em francês, a palavra \emph{séxe} e, em alemão, a palavra \emph{Geschlecht}, designam tanto diferenças físicas como psicológicas, sociais e culturais \cite{FRAISSE2001}. \citeauthoronline{MORAES1998} (\citeyear{MORAES1998}) reporta que, em francês, frequentemente utiliza-se \emph{rapports sociaux de séxe} ao invés de \emph{gendre} para se designar \emph{gênero}. Comparado com o termo inglês \emph{gender}, ``a palavra gênero, em português, é um substantivo masculino que designa uma classe que se divide em outras, que são chamadas espécies'', definição então retirada do Novo Dicionário Aurélio por \citeauthoronline{MORAES1998} (\citeyear{MORAES1998}, p.101).
Já hoje, em 2014, o Dicionário Aurélio
\footnote{Dicionário Aurélio Online: \url{http://www.dicionariodoaurelio.com/genero} Acesso em 26 de setembro de 2014.}
já comporta entre suas definições, aquela em que \emph{gênero} pode ser entendido como o ``conjunto de propriedades atribuídas social e culturalmente em relação ao sexo dos indivíduos''.
Porém, entre as definições mais gerais apresentadas pelo Dicionário Michaelis, \emph{gênero} é definido da seguinte forma
\footnote{Dicionário Michaelis Online: \url{http://michaelis.uol.com.br/moderno/portugues/index.php?lingua=portugues-portugues&palavra=g\%EAnero} Acesso em 26 de setembro de 2014}:

\begin{citacao}
s.m. (lat *\emph{generu}, por \emph{genus}) 1 Grupo de seres que têm iguais caracteres essenciais. 2 \emph{Lóg.} A classe que tem mais extensão e portanto menor compreensão que a espécie. 3 \emph{Biol.} Grupo morfológico intermediário entre a família e a espécie. 4 \emph{Gram.} Flexão pela qual se exprime o sexo real ou imaginário dos seres. 5 \emph{Gram.} Forma do adjetivo ou pronome com relação ao gênero dos nomes a que se refere. 6 Agrupamento de indivíduos que possuem caracteres comuns. 7 Espécie, casta, raça, variedade, sorte, categoria, estilo etc. 8 Qualidade, espécie, modo.
\end{citacao}

Percebe-se que o termo gênero designa um conceito em construção e consolidação, não apenas no Brasil, sendo necessário defini-lo sempre que o utilizarmos como denominação de categoria de análise \cite{MORAES1998}. Para isso, será feito um breve apanhado do surgimento e trajetória da palavra \emph{gender} ou \emph{gênero} nas pesquisas acadêmicas, inclusive no Brasil, bem como a evolução do conceito.

O conceito fundido e identitário de sexo e gênero, como se fossem sinônimos, pertenece a uma visão binária de mundo que define as mulheres mais próximas das natureza, do trabalho reprodutivo, da passividade e do irracional e, em oposição, define os homens mas próximos à cultura, ao trabalho produtivo, à ação e à racionalidade \cite{HARAWAY2004}.
Estudiosas feministas, rejeitando o determinismo bio-sexual para a situação social das mulheres, precisavam desmontar a naturalização das diferenças entre homens e mulheres que vinculava suas relações sociais, políticas e econômicas a seu aparelho reprodutor \cite{PISCITELLI2009}. 
Para \citeauthoronline{HARAWAY2004} (\citeyear{HARAWAY2004}, p.218), as feministas lutaram ``para remover as mulheres da caegoria da natureza e colocá-las na cultura como sujeitos sociais na história, cosntruídas e auto-construtoras''. Desssa forma, a evolução do conceito de gênero mescla-se à história do feminismo.

A chamada ``primeira onda feminista'' ocorreu entre o final do século XIX e o início do século XX nos países hoje considerados desenvolvidos da Europa e da Américado Norte. A principal bandeira reivindicada direitos iguais, compondo uma ideia de que deveria haver uma igualdade entre os sexos. Em decorrência dessa primeira movimentação, em diversos países, as mulheres conquistaram alguns direitos equivalentes aos dos homens, como o voto. Essa conquista do voto como um direito político caracterizou o movimento sufragista, que não pode ser confundido com o movimento feminista, embora seja parte dele.
A Filândia foi o primeiro país a garantir direito a votar e ser votado(a) igualmente a mulheres e homens, em 1906, quando ainda era um Principado do Império Russo \cite{RAY1918}.
Na Inglaterra, em 1865, John Stuart Mill apresenta ao Parlamento um projeto de lei dando o voto às mulheres, que não foi aprovado. Somente em 1928, o voto feminino é autorizado nas mesmas condições àsdos homens \cite{NELSON2004}.
Nos Estados Unidos, foi em 1920 aprovada a 19ª Emenda
\footnote{Fonte: \url{http://www.archives.gov/historical-docs/document.html?doc=13&title.raw=19th+Amendment+to+the+U.S.+Constitution:+Women\%27s+Right+to+Vote} Acesso em 02 de novembro de 2014.} que proíbia o estabelecimento de qualquer restrição ao voto (estadual e federal) baseadas no sexo do(a) votante. 
 
Na década de 1930, Mead, uma antropóloga estadunidense, problematiza a fixitude dos conceitos \emph{feminilidade} e \emph{masculinidade} a partir de uma pesquisa comparativa entre três sociedades tribais na Nova Guiné  \cite{MEAD2000}. A pesquisadora conclui não haver um temperamento inato, universal que tenha origem biológica, ligada ao aparelho reprodutor. Ela observa que traços de caráter são aprendidos em sociedade, podendo, portanto, ser modificados e até desaprendidos. Ela deixa legado teórico que suporta a ideia de que existe uma cosntrução cultural da diferença sexual.
%http://pt.scribd.com/doc/178229042/Resumo-Sexo-e-Temperamento-Margareth-Mead

Em 1949, a filósofa francesa Beauvoir lança a obra \emph{O Segundo Sexo}, considerado precursor da ``segunda onda feminista''\cite{PISCITELLI2009}. Ainda que \citeauthoronline{BEAUVOIR1967} (\citeyear{BEAUVOIR1967}) não cite o conceito de  ``papel social'' ou mesmo ``papel sexual'', ela enfatiza logo de início que o conceito do que é ser uma mulher é uma construção social:

\begin{citacao}
nenhum destino biológico, psíquico, econômico define a forma que a fêmea humana assume no seio da sociedade; é o conjunto da civilização que elabora esse produto intermediário entre o macho e o castrado que qualificam de feminino.
\cite[p.09]{BEAUVOIR1967}
\end{citacao}

Em sua obra, \citeauthoronline{BEAUVOIR1967} (\citeyear{BEAUVOIR1967}) tem por foco questionar a dominação masculina, sem deixar de questionar também a eficácia do movimento feminista forjado até então no combate a essa dominação. Ela julgava ser possível esse combate ser bem sucedido ser fossem combatidos elementos como: forma com que mulheres eram educadas; instituição de casamentos opressores; maternidade compulsória; vigência de um duplo padrão de moralidade sexual que permitiam maior liberade sexual somente aos homens; e falta de trabalhos dignos e bem remunerados que possibilitassem independência econômica às mulheres. 

Quase que concomitantemente, nos Estados Unidos, nasce um novo par de categorias de estudos, o sexo-gênero \cite{FRAISSE2001,STOLKE2004,HARAWAY2004}. A distinção entre as característica biológicas e as características sociais torna-se mais difundida, ou seja, na academia e na sociedade passa ser ser considerada a noção de que posturas sociais de identidade masculina ou feminina não estabelecem relação biunívoca com o sexo anatômico.
A nominação dessa construção cultural pela palavra gênero ocorre em 1958, na Califórnia, quando foi empreendida uma pesquisa acerca da identidade de gênero no \emph{California Gender Identity Center}. Os resultados foram apresentados pelo psicanalista Robert Stoller em 1963, no Congresso de Pscicanálise de Estocolmo. Essa mesma pesquisa embasou a elaboração do primeiro volume de \emph{Sex and Gender} de \citeauthoronline{STOLLER1968} (\citeyear{STOLLER1968}). Essa obra expôs o quanto a relação sexo e gênero não é automática, nem estrita, discorrendo ainda sobre casos em que a anatomia da genitália não seria compatível com a identidade masculina ou feminina da pessoa. Assim, Stoller formula um conceito de gênero ligado à cultura, enquanto o conceito de sexo permanece ligado à morfologia corporal.

Em 1970 e 1980, o debate sobre esse par de categorias (sexo-gênero) toma espaço na comunidade acadêmica estadunidense. A antropóloga \citeauthoronline{RUBIN1975} (\citeyear{RUBIN1975})  introduz a categoria gênero no debate sobre opressões sociais sofridas pelas mulheres por meio do seu ensaio \emph{The Traffic in Women: Notes on the 'Political Economy' of Sex}. Nessa obra, \citeauthoronline{RUBIN1975} faz uma análise marxista sobreposta ao sistema sexo-gênero da qual depreende que no sistema de trocas capitalista, os homens estabelecem-se como vendedores e as mulheres são estabelecidas como mercadorias para serem trocadas.
Rubin dialoga com Lévi-Strauss que aponta ser o casamento o dispositivo mais importante de aliança entre as famílias, inexistente senão fosse pelo \emph{tabu do incesto} \cite{STRAUSS2010}. Para \apudonline{RUBIN1975}{PISCITELLI2009} esse tabu é precedido por outro, o da \emph{homossexualidade}. Isso porque, mediante a divisão sexual do trabalho
\footnote{A expressão \emph{divisão sexual do trabalho} foi inicialmente utilizada por etnólogos para se referir à repartição das atividades entr homens e mulheres nas sociedades que estudavam \cite{KERGOAT2004}. Esta autora afirma ainda que ``a divisão sexual do trabalho é aquela decorrente das relações sociais de sexo'', o que será explorado mais adiante neste capítulo.} e ao tomar a menor unidade de sobrevivência econômica a família, tem-se necessariamente um homem e uma mulher, numa relação heterossexual de dependência mútua. Rubin discute também o trabalho doméstico, dando visibilidade a um trabalho que muitas vezes viabiliza o sustento do trabalhador (geralmente homem) sem que seja remunerada (a mulher). Por fim, ela consegue articular teoricamente gênero e sexualidade de forma que o conceito de gênero constituído até então não reside apenas em identificação com um determinado sexo, mas pressupõe que o desejo sexual seja por indivíduo do sexo oposto. 
% \url{https://ensaiosdegenero.wordpress.com/tag/gayle-rubin/} 
% \url{http://ensaiosdegenero.wordpress.com/2012/04/16/o-conceito-de-genero-por-gayle-rubin-o-sistema-sexogenero/}

A distinção entre sexo e gênero foi extremamente útil às feministas acadêmicas, pois sinalizava um lastro teórico para embasar os estudos sobre a condição da mulher, muitas vezes inferiorizada por sua condição biológica inerente. Com isso, o questionamento à lógica binária de interpretação do mundo passou a ser menos frequente e incisivo \citeauthoronline{HARAWAY2004} (\citeyear{HARAWAY2004}, p.218) e, porque não, superada em alguma medida. Conforme pode-se ver no trabalho de Rubin, o conceito de gênero foi além de separar dimensões culturais e biológicas de mulheres e homens. Cada vez mais o conceito de gênero passa a significar também a superação da leitura binária de mundo que só permite feminilidade ou masculinidade. Para \citeauthoronline{HEILBORN1992} (\citeyear{HEILBORN1992}, p.41):

\begin{citacao}
A categoria de gênero não deve ser acionada como um substituto de referência para homem ou mulher. Seu uso designa, ou deveria fazê-lo, a dimensão inerente de uma escolha cultural e de conteúdo relacional. Por outro lado, traz embutida a articulação desse código, que se apropria da articulação da diferença sexual tematizando-a em masculino e feminino, com outros níveis de significação dos universos.
\end{citacao}

Se a primeira onda do feminismo reinvindicou direitos iguais, a segunda onda avançou e lutou pelo exercícioigual dos direitos. Na primeira onda buscava-se provar que as diferenças entre o feminino e o masculino eram de origem social e não biológica. Tal afirmação não é abandonada na segunda onda, mas aprofundada, passando-se a buscar as origens de tais diferenças sócio-culturais. Nessa construção, segundo \citeauthoronline{PISCITELLI2009} (\citeyear{PISCITELLI2009}, p.133-134):

\begin{citacao}
A categoria ``mulher'' foi desenvolvida pelo feminismo da segunda onda em leituras segundo as quais a opressão das mulheres está além de questões de classe e raça, atingindo todas mulheres, inclusive as mulheres das classes altas e brancas. [...] O reconhecimento político das mulheres como coletividade ancora-se na ideia de que o que une as mulheres ultrapassa em muito as diferenças entre elas. Isso criava uma ``identidade'' entre elas.
\end{citacao}

Se essa uniformização entre as mulheres foi útil para forjar uma união na conquista por direitos, em meados da decada de 1970 e início dos anos 1980, já era questionada. Feministas negras e mulheres de países subdesenvolvidos \cite{FURTADO2009} cada vez menos identificavam-se com o arcabouço teórico hegemônico e homogêneo apresentado por feministas dos países do ``norte rico'', inclusive por Rubin. Assim, a ``terceira onda feminista'' desdobra-se em feminismos diversos. Afinal, as mulheres negras contam com trajetória histórica diferente das mulheres brancas, grande parte das vezes tendo a escravidão e suas consequências como parte determinante da vida de sua ancestralidade \cite{HOOKS1990,CRENSHAW2002}. No caso de países subdesenvolvidos, como o Brasil, não cabe comparar \emph{ipsis literis} a trajetória das mulheres (mesmo brancas) brasileiras com as europeias. A título de ilustração, o estudo de \citeauthoronline{PINTO2004} (\citeyear{PINTO2004}) apresenta como as mulheres brasileiras são vistas como mais maternais, com vocação para a domesticidade e muito mais ``racializadas'' do que as portuguesas.

%Mas, segundo \citeauthoronline{WIZEMAN2001}(\citeyearonline{WIZEMAN2001}) os termos sexo e gênero não são sinônimos e, conforme definição adotada pelo Instituto de Medicina da \emph{National Academy of Sciences} o sexo é uma classificação ``de acordo com os órgãos reprodutores e funções [biológicas] atribuídas pelo complemento cromossômico''. Gênero, por sua vez, é a ``auto-representação de um pessoa como masculino ou feminino, ou como a pessoa é percebida por instituições sociais com base na apresentação de gênero do indivíduo''.

Oferencendo alguma resposta a essas demandas por interseccionalidade
\footnote{Interseccionalidade, segundo \citeauthoronline{CRENSHAW2002} (\citeyear{CRENSHAW2002}, p.177) ``trata especificamente da forma pela qual o racismo, o patriarcalismo, a opressão de classe e outros sistemas discriminatórios criam desigualdades básicas que estruturam as posições relativas de mulheres, raças, etnias, classes e outras.''} em 1986, a historiadora pós-estruturalista Joan Scott publica seu artigo \emph{Gender: A Useful Category of Historical Analysis} em que faz uma leitura crítica da utilização do termo \emph{gênero} como categoria de análise e relaciona necessariamente esta categoria a outras como classe e raça, pois demonstra ser o gênero necessariamente imbricado a relações hierarquizadas de poder:

\begin{citacao}
a oposição binária e o processo social das relações de gênero tornam-se,
ambos, partes do sentido do próprio poder. Colocar em questão ou mudar um aspecto ameaça o sistema por inteiro. Se as significações de gênero e de poder se constróem reciprocamente, como é que as coisas mudam? [\ldots] o gênero tem que ser redefinido e reestruturado em conjunção com uma visão de igualdade política e social que inclui não só o sexo, mas também, a classe e a raça. \cite[p.1073,1075]{SCOTT1986}
\end{citacao}

%Assim como Scott, a filósofa estadunidense Judith Butler também tem influência foucaultiana e é pós-estruturalista. Em sua obra \emph{Gender Trouble: Feminism and the Subversion of Identity} publicada em 1990 Butler questiona a coerência entre sexo (biológico), gênero (construção cultural) e desejo (sexual). Para ela, existe uma regra tácita  heterossexual socialmente aceita como correta, estimulada, e que exige uma determinada coerência na tríade sexo-gênero-desejo. A partir dessa foram de ler o gênero, articulado ao desejo sexual, é que pessoas transgênero passam a ter algum arcabouço teórico que lhes abarque. \citeauthoronline{BUTLER1999} (\citeyear{BUTLER1999}) descreve a performatividade, logo, para ela, o gênero seria um ato intencional, performativo e que gera significados.

É então necessário olhar a construção das identidades de gênero à luz das relações de poder e olhar brevemente como se deu a evolução dos direitos, especialmente na sociedade brasileira. As mulheres no Brasil escravocrata dispunham de uma grande imobilidade geográfica e mesmo as mulheres das classes dominantes raramente saíam às ruas e, quando o faziam, nunca estavam desacompanhadas \cite{SAFIOTTI1976}. Mulheres e homens de então desfrutavam de maneira assimétrica do direito de ir e vir.

Na campo dos direitos políticos, o movimento sugfragista das brasileiras não teve tanta capilaridade nem foi um movimento de massas como nos Estados Unidos, Inglaterra ou Rússia. Ele teve início na década de 1910, quando o Partido Republicano Feminino é fundado no Rio de Jeneiro com o objetivo de instaurar o debate acerca do voto feminino 
\footnote{
Bertha Lutz, filha do cientista Adolfo Lutz, licenciou-se em Ciências Naturais na Sorbonne de Paris e, ao retornar ao Brasil, funda a Federação Brasileira pelo Progresso Feminino, em 1919, que leva adiante a luta pelo sufrágio feminino \cite{PINSKY2003}. A primeira cidade a autorizar o voto feminino em eleições foi Mossoró (RN), em 1928. Em nível nacional, Getúlio Vargas autoriza em 1931 o voto feminino apenas às mulheres solteiras, viúvas com renda própria ou casadas com a autorização do marido.}.
A igualdade de condições de voto entre homens e mulheres se concretiza em 1932, pelo Decreto nº 21.076 que autoriza o voto a qualquer cidadã ou cidadão com idade superior a 21 anos. A eleição de 1933 foi a primeira em que mulheres puderam participar do pleito, votando e sendo votadas, como Carlota Pereira Queiroz, a primeira deputada brasileira, que participou da Assembleia Nacional Constituinte entre 1934 e 1935 \cite{TABAK1989}.

Embora o direito ao voto tenha sido emblemático, a ideia de desfrutar de \emph{direitos iguais} na sociedade, mulheres e homens, tratava também de outros direitos como o acesso à educação e poder ter posse de bens - por muito tempo, de acordo com a lei, só homens podiam ser proprietários de casas, por exemplo \cite{PISCITELLI2009}. Subjacente a esses questionamentos das mulheres tecia-se o conceito de ``papel social'', bastante difundido a partir da década de 1930. Para \citeauthoronline{PISCITELLI2009} (\citeyear{PISCITELLI2009}, p.127), a teoria dos papeis sociais buscava:

\begin{citacao}
compreender os fatores que influenciam o comportamento humano. A ideia é que os indivíduos ocupam posições na sociedade, desempenhando papeis de filho, de estudante, de avô. [...] A ideia de posições ocupadas no desempenho dos papeis faz referência a categorias de pessoas qie são reconhecidas coletivamente. Um dos atributos que podem servir de base para a definição dessas categorias é a idade. [...] Outro desses atributos pode ser o sexo. Nesse caso, homens e mulheres desempenham papeis culturalmente construídos: os papeis sexuais.
\end{citacao}

Essa busca por um leque de direitos não foi um movimento só das mulheres, mas um movimento de luta por cidadania
\footnote{A cidadania para \cite{CARVALHO2002} é entendida como o exercício pleno de três direitos: direitos civis, direitos sociais e direitos políticos. Os civis são aqueles considerados direitos fundamentais, como o direito à vida, à liberdade, à propriedade, à igualdade perante a lei. Eles garantem a vida em sociedade e dependem da existência de uma justiça independente, eficiente, barata e acessível a todos. Os políticos se referem à participação do cidadão no governo da sociedade. Seu exercício é limitado a uma parcela da população definida por idade, por exemplo, e consiste na capacidade de fazer demonstrações políticas, de organizar partidos, de votar, de ser votado. Por fim, os sociais são aqueles que garantem a participação na riqueza coletiva e se baseia na ideia de justiça social. Incluem os direitos à educação, ao trabalho, ao salário justo, à saúde, à aposentadoria.}. Sob o ponto de vista de gênero e cidadania \citeauthoronline{BRITO2001} (\citeyear{BRITO2001}) relembra que o conceito clássico de cidadania, o grego, excluía mulheres e escravos. Ela pontua que ao longo da história as identidades de homens e mulheres foram construídas pressupondo uma dicotomia entre o âmbito público e o privado. \citeauthoronline{BLAY2001} (\citeyear{BLAY2001}) relata que até os anos 1960/1970 era um fator negativo para a mulher participar da vida pública.
A partir de 1970, com o movimento feminista passa a haver críticas e questionamentos quanto à natureza, à separação e à natural atribuição dessas esfera a um determinado sexo. Assim, elabora-se uma perspectiva de análise a partir do gênero, e não do sexo biológico, pois o conceito de gênero também compreende as dimensões social e política do termo.

A partir dos anos 1990 o uso da categoria gênero tornou-se mais frequentemente utilizada no Brasil e, cada mais influenciada pelas diversas escolas de psicanálise para explicar a produção e a reprodução da identidade de gênero do sujeito.
A psicanalista brasileira \citeauthoronline{KEHL1998} (\citeyear{KEHL1998}) embora não tenha como central esse debate, participa dele e em sua obra \emph{Deslocamentos do Feminino}: ao invés de apartar sexo de gênero, assinala que gênero é um conceito que inclui a dimensão biológica do sexo, não sem somar-lhe atributos que a cultura provê.
A partir de Scott é cada vez mais corrente incorporar a dimensão da política e do poder na composição do conceito de gênero, conforme explicita \citeauthoronline{MORAES1998} (\citeyear{MORAES1998}, p.100):

\begin{citacao}
A expressão relações de gênero, tal como vem sendo utilizada no campo das ciências sociais, designa, primordialmente, a perspectiva culturalista em que as categorias diferenciais de sexo não implicam o reconhecimento de uma essência masculina ou feminina, de caráter abstrato e universal, mas, diferentemente, apontam para a ordem cultural como modeladora de homens e mulheres. Em outra palavras, o que chamamos de homem e mulher não é o produto da sexualidade biológica, mas sim de relações sociais baseadas em distintas estruturas de poder.
\end{citacao}

Essas relações de poder incidem tanto sobre as relações que se desdobram no espaço público, quanto as do espaço privado. No espaço público, ou não-doméstico, tem relevância para o presente estudo o mercado de trabalho. Ao longo do tempo, a urbanização e a industrialização levaram à ampliação da classe média e ao crescimento do consumo no Brasil. As mulheres entraram também neste processo, embora a maior parte das trabalhadoras tenha sido absorvida, ao menos inicialmente, no setor de serviços e com enorme concentração nos empregos domésticos, de menor rendimento. Constata-se assim que existe no aqui uma divisão sexual do trabalho \cite{KERGOAT2004}, que tem por características a destinação prioritária dos homens à esfera produtiva e das mulheres à esfera reprodutiva. Essa forma de divisão pauta-se em dois princípios: o da separação e o da hierarquização. O princípio da separação explicita a ideia de que há ``trabalhos de mulheres'' e ``trabalhos de homens'' enquanto o princípio da hierarquização indica existir uma diferença de valoração entre o trabalho do homem (produtivo, mais valioso) e o da mulher (reprodutivo, menos valioso). \citeauthoronline{BLAY2001} (\citeyear{BLAY2001}, p.84) fala sobre a mulher brasileira:

\begin{citacao}
Até a década de 1960 a história, quando focalizava a mulher, atinha-se às supostas atividades femininas fundamentais, isto é, às de um ser apêndice da família. A historiografia simplesmente ignorava a participação feminina no mercado de trabalho, a enorme freqüência com que sustentavam economicamente a si e aos seus.
\end{citacao}

Ao longo do século XX, observou-se aumento de mulheres na população economicamente ativa brasileira \footnote{A população economicamente ativa é obtida pela soma da população ocupada e desocupada com 16 anos ou mais de idade. ``População ocupada'' compreende as pessoas que, num determinado período de referência, trabalharam ou tinham trabalho mas não trabalharam (por exemplo, pessoas em férias). ``População desocupada'' compreende as pessoas que não tinham trabalho, num determinado período de referência, mas estavam dispostas a trabalhar, e que, para isso, tomaram alguma providência efetiva nos últimos 30 dias (consultando pessoas, jornais, etc.). Fonte: IBGE - disponível em \url{http://www.ibge.gov.br/apps/snig/v1/?loc=0,355030&cat=118,119,1,2,-2,-3&ind=87} Acesso em 21 de novembro de 2014}
 (ver Gráfico \ref{graf:evolucao-pea}) e a maior frequência feminina em empregos de jornadas menores (ver Gráfico \ref{graf:percent-jornadas}). Fenômeno que permanece no século XXI de acordo com estudo da \cite{ABRAMO2010}: (i) observa-se que de 2001 a 2010 manteve-se a preponderância feminina em ocupações que demandam de 20 a 40h semanais; (ii) para homens, manteve-se o predomínio histórico de jornada superior a 40h semanais; (iii) o ingresso das mulheres no mercado de trabalho não alterou drasticamente o papel delas na família e, portanto, nas atividades ligadas às tarefas domésticas. Isto é, apesar de muitas mulheres terem entrado no mercado de trabalho algumas décadas atrás, elas ainda são responsáveis pela maior parte do trabalho doméstico. No Gráfico \ref{graf:jornadas-completas} é possível constatar não apenas essa divisão sexual do trabalho - produtivo, no mercado, e reprodutivo, no lar - mas também que as jornadas totais que acabam ficando a cargo da mulher são maiores. Os homens acumulam uma jornada de cerca de 50 horas por semana, as mulheres, 57 horas semanais.

\begin{grafico}[htb]%
    \caption{\label{graf:evolucao-pea}Percentual de indivíduos economicamente ativos, por sexo, no Brasil, entre 1950 e 2010}%
    \begin{center}%
        \includegraphics[width=1.05\textwidth]{./imagens/evolucao-pea1.png}%
    \end{center}%
    \fonte{Adaptado de \cite{ALVES2013}}
\end{grafico}%

\begin{grafico}[htb]%
    \caption{\label{graf:percent-jornadas}Percentual de trabalhadores(as) com jornadas de trabalho semanal acima de 44 horas e 48 horas e abaixo de 35 horas, por sexo, no Brasil, em 2008}%
    \begin{center}%
        \includegraphics[width=0.9\textwidth]{./imagens/jornada-muler2003.png}%
    \end{center}%
    \fonte{\apud[p.01]{PNAD2008}{OIT2008}}
\end{grafico}%

\begin{grafico}[htb]%
    \caption{\label{graf:jornadas-completas}Jornadas Médias para o Mercado de Trabalho e para Reprodução Social, por sexo, raça/cor e região geográfica, no Brasil, em 2003}%
    \begin{center}%
        \includegraphics[width=1.0\textwidth]{./imagens/jornadas-totais.png}%
    \end{center}%
    \fonte{\apud{PNAD}{SOARES2003}}
\end{grafico}%

%Mais um parágrafo sobre a divisão sexual do trabalho, dos papeis sociais - citar hirata aqui!

Esse é o ponto em que a atuação no espaço público e a no espaço privado vincula-se.
A mulher passa a poder desempenhar atividades antes tidas como ``masculinas'', porém sem ser desonerada de desempenhar as atividades tidas como ``femininas'', pois  ainda ``persistem nichos onde vigora uma imagem feminina vinculada à maternidade e ao cuidado da família, à saúde da prole'' \cite[p.94]{BLAY2001}. 
Assim, a ampliação do leque de papéis sociais que a mulher desempenha impacta as relações de poder dentro do ambiente doméstico, dentro da família. Isso molda as necessidades, interesses, atividades e padrão de viagens dos integrantes da família, a partir das identidades de gênero constituídas, forjadas pelos comportamentos dos indivíduos e da relação de poder estabelecida entre eles.

\clearpage
\section{Mobilidade Urbana}
% META: 10p.

\hl{terminar até 29 de novembro}

A palavra mobilidade, de acordo com o dicionário Michaelis
\footnote{Definição de mobilidade de acordo com o Michaelis disponível em \url{http://michaelis.uol.com.br/moderno/portugues/index.php?lingua=portugues-portugues&palavra=mobilidade} Acesso em 06 de novembro de 2014.}
significa ``(i) propriedade do que é móvel ou do que obedece às leis do movimento;
(ii) deslocamento de indivíduos, grupos ou elementos culturais no espaço social;
(iii) movimento comunicado por uma força qualquer;
(iv) falta de estabilidade, de firmeza ou inconstância''.
Tal definição reflete toda uma gama de conceitos relacionados a movimento e/ou deslocamento, o que na área de transportes relaciona-se imediatamente a viagens. No Brasil, há cerca de 100 anos atrás a maior parte das viagens era a pé
\footnote{Os primeiros carros foram montados em São Paulo pela Ford na década de 1910. Fonte: \url{http://www.carroantigo.com/portugues/conteudo/curio_hist_carro_brasileiro.htm} Acesso em 25 de outubro de 2014} 
e, quando muito, feitas por tração animal (cavalo ou boi).
Isso incorria em necessárias baixas velocidades de deslocamento e grande parte das pessoas acabavam por desenvolver suas atividades nas proximidades de onde nasceram por toda a vida. Nesses últimos dois séculos o cenário mudou bastante, as mais diversas tecnologias se desenvolveram, os rendimentos aumentaram e a mobilidade aumentou \cite{METZ2011, p.6} – como exemplos icônicos dessa maior mobilidade figuram a utilização do carro e do avião.

\hl{XXXXXX}

 englobando, por exemplo, movimentos migratórios entre países ou dentro de uma mesma nação. Outro uso do termo é ligado ao intercâmbio de estudantes e pesquisadores de diferentes instituições de origem – o que dá origem à expressão \emph{mobilidade acadêmica}. Ademais, há um olhar sobre a mobilidade que teriam condições geo-demográficas como elementos de contorno, delimitando assim as áreas da mobilidade rural e urbana.
 
Embora pareçam óbvias à primeira vista – porque em alguma medida vividas – as diferenças entre rural e urbano são bem menos claras quando olhadas mais de perto. No Brasil as distinções nascem de critérios político-administrativos, originados em decreto de 1938
\footnote{Decreto-Lei nº 311, de 2 de Março de 1938 disponível em: \url{http://www2.camara.leg.br/legin/fed/declei/1930-1939/decreto-lei-311-2-marco-1938-351501-publicacaooriginal-1-pe.html} Acesso em 06 de novembro de 2014}
de Getúlio Vargas e, até hoje, no Brasil, baseia-se em critérios políticos administrativos. Segundo o \cite{IBGE2001, p.63} “como situação urbana consideram-se as áreas correspondentes às cidades (sedes municipais), às vilas (sedes distritais) ou às áreas urbanas isoladas“. Como rural, classifica-se tudo o que não se configure urbano. Trata-se de definição legal (jurídica) ocorrida frequentemente no campo da política e passível de críticas, como a de \cite{GRABOIS2001} que aponta que tal classificação não considera as diferentes funções dos aglomerados como critério. A questão da definição do que é rural vai além da abordagem teórica e tem como pano de fundo as diferenças de tributação entre as áreas rural e urbana. Como saída a essa arbitrariedade que fica a cargo dos poderes municipais, \cite{VEIGA2002} elenca três critérios que entende importantes a se considerar nesse tipo de classificação: (i) população total do município, (ii) densidade demográfica e (iii) localização.
O objeto deste trabalho é RMSP, composta por 39 municípios
\footnote{Os 39 municípios que compõem a RMSP são agrupados em 6 regiões de acordo com Lei Complementar estadual nº 1.139, de 16 de junho de 2011. Na região central está o São Paulo. Na região sudoeste encontram-se 8 municípios, a saber, Juquitiba, São Lourenço da Serra, Embu-Guaçu, Itapecerica da Serra, Embu, Tabão da Serra, Cotia e Vargem Grande Paulista. Na Região Oeste encontram-se 7 municípios, a saber, Pirapora do Bom Jesus, Santa de Parnaíba, Barueri, Jandira, Itapevi, Carapicuiba, Osasco. Na região norte encontram-se 5 municípios, a saber, Cajamar, Caieira, Franco da Rocha, Francisco Morato, Mairiporã. Na região leste encontram-se 11 municípios, a saber, Santa Isabel, Arujá, Guarulhos, Itaquaqueceteuba, Guararema, Poá, Suzano, Ferraz de Vasconcelos, Mogi das Cruzes, Biritiba Mirim, Salesópolis. Na região sudeste encontram 7 municípios, a saber, Santo André, São Bernardo do Campo, São Caetano do Sul, Diadema, Ribeirão Pires, Rio Grande da Serra.}, todos, atualmente, contam com áreas urbanas. 
Muito embora se vá seguir as classificações oficiais do IBGE e do Metrô-SP, entende-se como salutar esta breve discussão sobre o significado do que é ser área urbana ou rural. Ou seja, ressalta-se que “o espaço rural tem passado recentemente por um conjunto de mudanças com significativo impacto sobre suas funções e conteúdo social” \cite{(MARQUES2002, p. 96)} e mesmo na área de enfoque, urbana, encontram-se atividades agrícolas (comumente tidas como rurais), afinal, são cada vez mais imprecisos os limites entre um e outro \cite{(MINGIONE1987)}. Um outro fenômeno que merece alguma atenção, dada a natureza deste trabalho, é o êxodo rural seletivo que vem sendo constatado por alguns pesquisadores. \cite{(RAUBER2010)} constata no Rio Grande do Sul que a emigração do campo é desigual em gênero e em idade: mulheres e jovens migram mais, homens e idosos são os que permanecem no campo, nas atividades rurais. Fenômeno semelhante é constatado em Santa Catarina
\footnote{Êxodo seletivo é retratado em Santa Catarina pelo documentário ``Celibato no Campo'' de Cassemiro Vitorino e Ilka Goldschmidt, 2013, disponível em \url{http://www2.camara.leg.br/camaranoticias/tv/materias/OLHARES/440520-CELIBATO-NO-CAMPO.html} Acesso em 15 de outubro de 2014.},
e em alguns países europeus
\footnote{Relatório do Parlamento Europeu em 2003 apontava que somente 37\% da mão-de-obra rural da União Europeia era de mulheres. Disponível em \url{http://www.europarl.europa.eu/sides/getDoc.do?type=REPORT&reference=A5-2003-0230&format=XML&language=PT} Acesso em  16 de outubro de 2014}. Ou seja, existe diferença relacional entre a mobilidade feminina e a masculina expressa, por exemplo, nos deslocamentos campo-cidade.
O recorte deste trabalho, entretanto, concentra-se majoritariamente nos deslocamentos intra-urbanos, expressos amplamente pelo conceito de mobilidade urbana. 

% Para \cite{MARQUES2002} os espaços urbano e rural “são pensados como segmentos de uma totalidade dialética, ou seja, totalidade cuja unidade se forma na diversidade”. 

Acessibilidade


``o estudo dos problemas urbanos é indissociável da relação campo-cidade. Por isso, o ofoc da an[alise, independente da época estudada, precisa abranger essa relação.'' \cite[p.154]{FREITAG2007}

A mobilidade urbana é um elemento fundamental para que seja possível garantir aos habitantes acesso aos bens que orferece \cite{IEMA2010}.

FALAR DE ACESSIBILDIADE e sua diferença de MOBILIDADE
\begin{citacao}
A acessibilidade em geral é medida pela quantidade e/ou diversidade de destinos quea pessoa consegue alcançar, por certa forma de transporte, em determinado tempo. Quanto maior for esta quantidade, maior é a acessibilidade, ou seja, mais oportunidades as pessoas terão para realizar atividades desejadas ou necessárias. \cite[p.42]{VASCONCELLOS2012}
\end{citacao}

\clearpage
\section{Sustentabilidades}
% META: 10p.

Por definição de sustentabilidade encontram-se nos dicionários descrições bastante simples e amplas, que podem ser resumidas como ``a qualidade de ser sustentável'' \cite{MICHAELIS2014}, o que posterga a dúvida para a questão: o que é ser sustentável? Segundo \citeauthoronline{BLACK2010} (\citeyear{BLACK2010}), é aquilo que pode ser mantido ou que dure. É evidente que tal durabilidade não é eterna, mas por um determinado período. A ideia da permanência leva a crer que tal período seja longo, que relacione-se à perspectiva de longo prazo, mas de quão longo se trata, é uma indefinição, até hoje. Hoje, inclusive, é cada vez mais frequente o uso do termo sustentável como modificador ao invés de sustentabilidade como um conceito fechado em si. Aqui, exploraremos o desenvolvimento  sustentável e o transporte sustentável como as ``sustentabilidades'' de interesse.

As primeiras preocupações e dicotomizações entre desenvolvimento e meio ambiente remontam o fim da década de 1960, com o Clube de Roma
\footnote{O Clube de Roma fora fundado em 1968 por Aurelio Peccei e Alexander King e que consistia num grupo de pessoas ilustres (empresários, líderes religiosos, políticos, entre outros) que se reuniam para discutir assuntos ligados à política, economia e, também, meio ambiente. Para saber mais: \url{http://www.clubofrome.org/} Acesso em 06 de novembro de 2014.}, que será o berço da obra \emph{The Limits to Growth}. Este livro, publicado em 1972, problematiza pela primeira vez a questão do crescimento exponencial \emph{versus} a finitude dos recursos disponíveis e também se propõe a simular e tentar prever as consequências da intereção antrópica com sistemas não-antrópicos \cite{MEADOWS1972}. Nesse mesmo ano, ocorre a Conferência sobre o Ambiente Humano das Nações Unidas em Estocolmo.

Em 1983, as Nações Unidas (ONU) fundam a Comissão Mundial sobre Meio Ambiente e Desenvolvimento (WCED), composta por 19 delegados de 18 países, com a missão de produzirem um estudo sobre desenvolvimento em escala global, considerando aspectos como sustentabilidade e meio ambiente num perspectiva de longo prazo. Assim, a expressão ``desenvolvimento sustentável'' aparece pela primeira vez em \citeyear{WCED1987}, no relatório \emph{Our Common Future} da WCED, também conhecido como \emph{Brundtland Report} 
\footnote{Gro Harlem Brundtland era o Primeiro Ministro da Noruega e foi quem comandou a Comissão Mundial sobre Meio Ambiente e Desenvolvimento (WCED). Fonte: \url{http://www.un-documents.net/our-common-future.pdf} Acesso em 06 de novembro de 2014.}, onde é apresentado o clássico conceito:

\begin{citacao}
desenvolvimento sustentável é aquele que satisfaz as necessidades do presente sem comprometer a capacidade das gerações futuras satisfazerem as suas próprias necessidades. Este conceito contém em si outros dois conceitos-chave: o de ``necessidades'', em particular as necessidades essenciais dos pobres do mundo, às quais deve ser dada prioridade absoluta; e a ideia de limitações impostas pelo estágio tecnologico e de organização social sobre a capacidade do meio ambiente de satisfazer as necessidades presentes e futuras. 
\cite[p.41]{WCED1987}
\end{citacao}   

Essa definição consolidou-se na Conferência das Nações Unidas sobre Meio Ambiente e Desenvolvimento ocorrida em 1992 no Rio de Janeiro, também conhecida como ECO-92. Em quase vinte anos, o conceito popularizou-se, ganhou robustez e também ficaram mais nítidas suas limitações de implementação. Um relatório de balanço da ONU publicado em 2010 \cite{ONU2010} indica haver convergência conceitual de que o desenvolvimento sustentável está alicerçado sobre três pilares: desenvolvimento econômico, equidade social e proteção ambiental. O mesmo documento reconhece, porém, que apesar de visionário e integrador, o conceito tem se mostrado de difícil implementação pelos países e pouco tem sido abraçado em sua completude pelas políticas das diversas nações. 

Embora o desenvolvimento sustentável pretenda englobar os três pilares, o desenvolvimento é frequentemente sinônimo de desenvolvimento econômico e a sustentabilidade fica muitas vezes compartimentada à questão ambiental \cite{ONU2010}. Um dos caminhos que vem sendo cosntruído para tentar superaressa dificuldade é tornar o conceito menos difuso e mais palpável por meio de indicadores \cite{CHAMBERS2000,BOULANGER2008,BARRETT2010,FORTES2012} e metas (de preferência quantitativas) a serem atingidas num determinado prazo \cite{ONU2010,ONU2014}.

Outra saída, não excludente com esta recém apresentada, é adicionar outros pilares na conceituação do que seja desenvolvimento sustentável, como faz 
\citeauthoronline{BANISTER2005} (\citeyear{BANISTER2005}). Ele elenca outros dois fatores como fundamentais: (i) participação e (ii) governança. A dimensão da participação refere-se a envolver todas pessoas interessadas e envolvidas no processo, a saber, indivíduos, empresas, indústrias e governos. Argumenta que criar excluídos do processo torna muito mais difícil desenvolver as estratégias necessária de mudança. A dimensão da governança incide diretamente no processo de tomada de decisão, logo, significa mudanças nas estruturas organizacionais para que sejam facilitadas decisões intersetoriais.

\citeauthoronline{BANISTER2005} (\citeyear{BANISTER2005}) destaca cinco lições aprendidas desde o \emph{Brundtland Report} a partir das experiências de sucessos e fracassos: (i) as reduções devem começar modestamente; (ii) desestímulo às emissões de dióxido de carbono (CO$_2$) deve contar com mecanismos fiscais; (iii) deve haver incentivos fortes em pesquisa e desenvolvimento em ciência e tecnologia na temática das mudanças climáticas; (iv) embora todos países devam contribuir para a diminuição de emissões de carbono, a liderança cabe às nações mais ricas; (v) é preciso ação imediata e incerteza não é uma boa razão para inação ou atitudes fracas.
Ao se falar em sustentabilidades, fala-se necessariamente de mudanças de paradigma, profundas, e que podem até mesmo ser inatingíveis \cite{GLASBY2002}, embora possam ser perseguidas. \citeauthoronline{ROGERS2000} (\citeyear{ROGERS2000}) aponta que essa mudança deve necessariamente compreender as ``relações entre cidadãos, serviços, políticas de transporte e geração de energia, bem como seu impacto total no meio ambiente local e numa esfera geográfica mais ampla'' e aponta ainda as cidades, pensadas como organismos vivos, cujo metabolismo é bastante linear, precisando tornar-se mais circular (ver figura \ref{fig:cidade-metabolismos}).

\begin{figure}[htb]%
    \caption{\label{fig:cidade-metabolismos}Cidades de metabolismo linear e circular}%
    \begin{center}%
        \includegraphics[width=0.80\textwidth]{./imagens/richard-linear-circular.jpg}%
    \end{center}%
    \fonte{\cite[p.31]{ROGERS2000}}
\end{figure}%

Do ponto de vista econômico, a obra \emph{Limits to Growth} fez escola e trouxe à baila a hipótese de que o crescimento econômico teria um teto, função dos recursos (naturais) disponíveis. Contudo, há estudiosos que se opõem a isso, como \citeauthoronline{KRUGMAN2014} (\citeyear{KRUGMAN2014}) em seu recente artigo \emph{Slow Steaming and the Supposed Limits to Growth}. O economista argumenta que é possível manter o crescimento econômico real (do PIB) e ainda assim reduzir a emissão de gases do efeito estufa. Para chegar a essa conclusão ele apresenta uma demonstração - bastante simplista - que considera o consumo de energia dos navios em função de suas velocidades e conclui ser possível manter um caudal econômico constante e, concomitantemente, diminuir o consumo de energia do sistema.

Entre as principais preocupações no âmbito ambiental estão a diminuição das reservas de petróleo, aquecimento global por conta da emissão de gases do efeito estufa, poluição (atmosférica, sonora e hídrica) e presença de chuva ácida. Diversos estudos indicam que exite correlação entre a ocorrência de diversos tipos de doenças (cardiorrespiratórias, câncer, entre outras) e a exposição alguns poluentes presentes na atmosfera \cite{WHO2000,WHO2006,BRUNEKREEF2012,MIRANDA2012}. Em São Paulo, \citeauthoronline{GOUVEIA2006} (\citeyear{GOUVEIA2006}) observaram associação estatisticamente significante entre o aumento no nível de poluentes na atmosfera e o aumento de hospitalizações por causas diversas, em todos grupos etários estudados. As emissões de CO$_2$, indicado como o principal gás responsável pelo efeito estufa, aumentaram cerca de 60\% e a parcela cuja origem são os sistemas de transporte também aumentou de 19,3\% para 28,9\% entre 1971 e 2001 \cite{BANISTER2005}.

Sob o prisma da equidade social, o acesso equânime a oportunidades de educação, trabalho, saúde e lazer é um dos pontos centrais. A equidade, associada à ideia do ``ser justo'', inevitavelmente referir-se-á à distribuição social de custos e benefícios , bem como em que grau essa distribuição é considerada adequada e que corrobore para a promoção da justiça \cite{LITMAN2006}. Aqui também o transporte tem papel estruturador já que pode ser o elemento que provê ou barra o acesso às oportunidades. \citeauthoronline{SANCHEZ2003} (\citeyear{SANCHEZ2003}) já apontava que equidade seria um dos temas estratégicos nas políticas de transportes. As megalópoles latino-americanas são, por vezes, cidades ``partidas'' \cite{VENTURA2001} entre a ``legal'' e a ``real'' \cite{ALVA1997}
\footnote{Estima-se que cerca de 40\% ou mais da população possui moradia em condição irregular \cite{FREITAG2007}.},
onde as vias de circulação frequentemente são cicatrizes no tecido urbano - por exemplo, em São Paulo, o ``Minhocão'' \cite{ABASCAL2010}, os monotrilhos \cite{ROLNIK2010} ou mesmo uma rua na favela de Paraisópolis (ver Figura \ref{fig:paraisopolis}), em São Paulo.

\begin{figure}[htb]%
    \caption{\label{fig:paraisopolis}Favela de Paraisópolis: sua parca arborização e a divisa com parte nobre do bairro Morumbi em São Paulo}%
    \begin{center}%
        \includegraphics[width=1.0\textwidth]{./imagens/paraisopolis.jpg}%
    \end{center}%
    \fonte{Foto da esquerda de Gustavo Roth; foto da direita de Tuca Vieira/Folha Imagens, disponível em: \url{http://www.scielo.br/scielo.php?script=sci_arttext&pid=S0103-49792010000200005} Acesso em 11 de novembro de 2014}
\end{figure}%

Pode-se observar nos três pilares clássicos da sustentabilidade o papel relevante dos transportes. \citeauthoronline{VASCONCELLOS2012} (\citeyear{VASCONCELLOS2012}) aponta ainda que a energia gasta na mobilidade por habitante de uma cidade, ou seja, quanta energia os moradores de um município precisam para deslocar-se permite ter uma ideia do ``grau de sustentabilidade'' da mesmo. Dessa maneira, cabe uma breve discussão sobre transporte sustentável.

Para Black, um transporte sustentável seria aquele que atende às ``atuais necessidades de transporte e mobilidade não devem comprometer a capacidade das futuras gerações satisfazerem as suas próprias necessidades'' \cite[p.151]{BLACK1996}, e também que ``provê transporte e mobilidade com combustíveis renováveis, minimizando as emissões prejudiciais ao ambiente local e globalmente, e prevenindo fatalidades, lesões e congestionamentos desnecessários'' \cite[p.12]{BLACK2010}.

Banister (\citeyear{BANISTER2005,BANISTER2008}) aponta algumas medidas a serem perseguidas para que se possa alcançar um transporte sustentável:
\begin{compactitem}[]
\item (i) reduzir a necessidade de viajar;
\item (ii) encorajar a troca para modos de transporte coletivo ou não motorizado;
\item (iii) reduzir o comprimento das viagens;
\item (iv) incentivar a adoção de sistemas e tecnologias de transporte mais eficientes, tanto para carga quanto para passageiros;
\item (v) reduzir a utilização de carros e caminhões de carga nas áreas urbanas;
\item (vi) reduzir, na fonte, ruídos e emissões dos veículos,
\item (vii) incentivar a utilização mais eficiente e ambientalmente consciente do estoque de veículos;
\item (viii) melhorar a segurança de pedestres e de todos usuários das (rodo)vias;
\item (ix) melhorar a atratividade das cidades para seus moradores, trabalhadores, compradores e visitantes.
\end{compactitem}

Embora cientes (organismos internacionais, governos e comunidades científicas) de medidas que corroborariam para o estabelecimento de um transporte sustentável, os padrões de mobilidade observados indicam uma dependência cada vez maior do automóvel (com poucas exceções), seja nos países desenvolvidos \cite{BANISTER2005}, seja nos países em desenvolvimento \cite{VASCONCELLOS2012}. \citeauthoronline{BANISTER2005} (\citeyear{BANISTER2005}) informa que entre 1984 e 1994 houve um aumento de 31\% na posse de veículos e que estimava-se chegar a 50\% em 2020. Ele também indica que a maior aprte da forta (70\% em 2005) encontrava-se nos países desenvolvidos.

Entretanto, isso não significa que a posse de carros não esteja crescendo nos países em desenvolvimento. No Brasil, a frota de automóveis vem crscendo desde 1960 conforme pode ser observado na Tabela \ref{tab:venda-veic-br}, sendo que em 2009, 53\% da frota total de veículos era composta por automóveis \cite{VASCONCELLOS2012}. Esse fato somado ao de que o automóvel é o modo que apresenta o maior consumo energético (ver Tabela \ref{tab:gep-modo}) levam a concluir que os o desenvolvimento no Brasil também trilha o caminho da insustentabilidade.

\clearpage
\begin{table}[htb]
    \IBGEtab{%\renewcommand{\arraystretch}{1.5}%%\ABNTEXfontereduzida%
	    \renewcommand{\arraystretch}{1.5}
        \caption{Venda interna de veículos no Brasil entre 1960 e 2009}
		\label{tab:venda-veic-br}
    }{%
	    \begin{tabular}{p{2.00cm} P{4.0cm} P{4.0cm} P{4.0cm}}
            \toprule
	           \headerTabCenterCell{Ano} &
		       \headerCell{Autos} &
		       \headerCell{Total} &
		       \headerCell{Fator de crescimento (total)} \\
		    \midrule \midrule
		        1960&
		        40.980&
		        131.499&
		        1\\
		    \midrule
		        1970&
		        308.024&
		        416.704&
		        3,2\\
		    \midrule
		        1980&
		        739.028&
		        980.261&
		        7,5\\
		    \midrule
		        1990&
		        532.906&
		        712.741&
		        5,4\\
		    \midrule
		        2000&
		        1.176.774&
		        1.489.481&
		        11,3\\
		    \midrule
		        2009&
		        2.474.764&
		        3.141.240&
		        23,9\\
		    \bottomrule
		\end{tabular}
    }{%
		\fonte{Adaptado de \cite[p.29]{VASCONCELLOS2012}}
		}
\end{table}

\begin{table}[htb]
    \IBGEtab{%\renewcommand{\arraystretch}{1.5}%%\ABNTEXfontereduzida%
	    \renewcommand{\arraystretch}{1.5}
        \caption{Consumo energético teórico dos modos de transporte em lotação plena}
		\label{tab:gep-modo}
    }{%
	    \begin{tabular}{p{4.00cm} P{4.0cm}}
            \toprule
	           \headerTabCenterCell{Modo de Transporte} &
		       \headerCell{gramas equivalentes de petróleo para mover um passageiro por um quilômetro}\\
		    \midrule \midrule
		        ônibus comum&
		        4,1\\
		    \midrule
		        metrô&
		        4,3\\
		    \midrule
		        motocicleta&
		        11,0\\
		    \midrule
		        automóvel&
		        19,3\\
		    \bottomrule
		\end{tabular}
    }{%
		\fonte{Adaptado de \cite[p.84]{VASCONCELLOS2012}}
		}
\end{table}

Se parece ilógica e insustentável a adoção do automóvel particular como modo principal de locomoção, por que ele continua tão bem cotado? A resposta a essa pergunta parece ser uma soma de fatores que o leva a ser um ícone, culturalmente simbólico e economicamente valorizado. 
Sobre o caráter simbólico, \citeauthoronline{BANISTER2005} (\citeyear{BANISTER2005},. p.05) expõe que o carro é visto como ``seguro, sempre disponível e nunca muito longe do seu motorista''. \citeauthoronline{URRY2001} (\citeyear{URRY2001}) indica ainda outros fatores que contribuem para esse \emph{status} do carro: 
\begin{compactitem}[]
\item (i) como um objeto manufaturado, nascido com o fordismo, é um ícone do sucesso capitalista; 
\item (ii) depois da moradia, é o principal bem de consumo que confere \emph{status} social ao indivíduo;
\item (iii) é um objeto de suficiente complexidade que sintetiza e ilustra um avanço tecnológico;
\item (iv) confere mobilidade individual e, portanto, liberdade para algumas escolhas como horários de saída e rotas adotadas;
\item (v) é revestido de um discurso ela mídia e pela indústria cultural que o liga ao sucesso e ao progresso.
\end{compactitem}

Sob a perspectiva econômica, na indústria brasileira, o ramo automobilístico tem tido um papel bastante central. Nos anos 1950 foram instaladas as três maiores montadoras à época na região de São Bernardo (SP), o que gerou emprego, aqueceu a indústria e também estimulou o nascimento de toda uma geração de motoristas de carro. Desde então, há uma pressão crescente por por mais vias, maiores, melhores e mais fluidas.
%Em 1990, com a estabilização da economia e o controle da inflação, houve o fortalecimento do setor da construção civil e a popularização do financiamento de motos e carros. Esses elementos geraram uma megalópole com, por exemplo, \emph{shopping centers} que dispõem de gigantesca quantidade de vagas de estacionamento e condomínios com pelo menos uma vaga de garagem por apartamento, sem que tudo isso seja devidamente comportado pelos espaços de circulação.
Dado que o espaço é finito, ao aumentar os espaços de circulação, diminuem-se os espaços disponíveis para abrigarem as atividades das pessoas. Há mais de 50 anos \apudonline[p.63]{OWEN1956}{BLACK2010} já conseguia reconhecer que:

\begin{citacao}
O problema do congestionamento se tornou tão grande que muitas comunidades estão chegando à conclusão de nunca haverá avenidas nem vagas de estacionamento suficientes que permitam o movimento de todas pessoas em carros particulares.
\end{citacao}

A RMSP sofre das contradições de políticas que apontam para direções diferentes, quando não antagônicas. No âmbito do município de São Paulo, conta-se com a ``Lei de Mudanças Climáticas'' \cite{LEICLIMASP2009} que prevê a redução de 30\% nas emissões dos gases do efeito estufa, além de substituição integral do uso de combustíveis fósseis por renováveis na frota de transporte público. No âmbito estadual, o Plano de Controle de Poluição Veicular 2011-2013 \cite{PCPV2011} indica, entre outros objetivos, a adoção da inspeção ambiental de veículos, uma (única) medida que incide sobre o transporte privado individual. Já no âmbito federal, para garantir aquecimento econômico e minimizar a taxa de desemprego, o Imposto sobre Produtos Industrializados (IPI) dos carros nacionais novos 1.0 foi a zero no primeiro semestre de 2012, sendo que até o final de 2014 não terá retornado ao patarmar dos 11\% \cite{FAZENDA2014}.

Em alguma medida o conjunto das políticas públicas transparece um desejo de não restringir a posse do carro, mas seu uso. Ou seja, deseja-se ao mesmo tempo desviar do impacto econômico que uma diminuição das vendas de carros geraria e regular o uso dos automóveis. Essa é a abordagem liberal que diversas cidades, de vários países do mundo vem adotando. Isto é, não se deseja impor restrições legais ou econômicas, mas entender e estimular comportamentos mais interessantes para o conjunto da sociedade e que corrobore para a construção de cidades mais sustentáveis. Todavia, \apudonline[p.7]{GILBERT2000}{BANISTER2005} deixa o alerta de que ``há uma ligação entre a posse do carro e uso do carro, e qualquer estratégia coerente para reduzir o uso do carro está fadada ao fracasso se realmente não abordar a causa da mobilidade insustentável, ou seja, o carro''.

%Dessa maneira, recai-se na necessidade de articulação entre planejamento urbano, ambiental e econômico.


\clearpage
\section{Intersecções e Sobreposições}
% META: 10p.

\hl{inserir texto que ficou no comp do Metrô em 24 denov}

\hl{finalizar este trecho no fds de 29 de dezembro}

%viagens mais curtas: \cite{HOWE1982,FAGNANI1983,MARAFFA1985,LAW1999a,IBIPO1992}

%Fazer a ressalva:
%Vale lembrar que o gênero concerne tanto a homens como mulheres, ainda que a maior parte das análises que se valham dessa categorias refiram-se às mulheres \cite{MORAES1998}

%Quando existe a posse do automóvel este fica mais frequentemente à disposição dos homens do que das mulheres.
%As mulheres ao andarem de automóvel são, com maior frequência, passageiras, remetendo à ideia de “não andar desacompanhada fora de casa”. \citeauthoronline{FOX1983}(\citeyear{FOX1983}) já indicava os padrões de viagens das mulheres: elas fazem menos viagens, viagens mais curtas e rápidas. Além disso, usam menos o automóvel e mais o transporte público.


%Ao estudar o comportamento da mulher em relação ao do homem nos transportes são frequentes duas abordagens na literatura \cite{BEST2005}: a primeira considera principalmente as diferenças de gênero decorrentes do mercado de trabalho \cite{HANSON1985} e a segunda prioriza as diferenças decorrentes do medo feminino em resposta à violência masculina \cite{TRENCH1992}.
%O presente trabalho se aterá à primeira vertende de pesquisa, investigando indução de comportamentos tidos como femininos e como masculinos a partir de variáveis sócio-econômicas. Não serão considerados efeitos comportamentais decorrentes da violência, o que não significa dizer que tais aspectos sejam menos relevantes ou que não possam ser explorados em análises futuras.

%Nos últimos 20 anos o assunto gênero e transporte tem atraído atenção crescente da comunidade científica. Pesquisadores começaram a examinar os padrões de mobilidade com o recorte de gênero considerando que há acesso desigual a recursos materiais e diferenças na escolha modal. São comuns duas abordagens, a primeira que considera principalmente as diferenças de gênero decorrentes do mercado de trabalho (Hanson e Johnston, 1985) e a segunda que prioriza as diferenças decorrentes do medo feminino da violência masculina (Trench et al., 1992). O escopo deste trabalho se limita à primeira abordagem.

%Embora a divisão de trabalho por gênero seja identificada como um fator que influencia a mobilidade, costuma-se ver o trabalho doméstico como uma restrição na participação do mercado de trabalho e o transporte decorrente desta atividade. Subestima-se o efeito do arranjo familiar no padrão de atividades femininas e as viagens geradas a partir de demandas domésticas. Se no mercado de trabalho vem sendo traçado um caminho que tende a diminuir o desequilíbrio de gênero, no trabalho doméstico ainda é a mulher a grande responsável pela sua execução.

%A participação no mercado de trabalho aumenta o uso do carro para ambos os gêneros, especialmente quando se trata de meio-período. Ao diminuir entre homens e mulheres a diferença na participação no mercado de trabalho diminui-se também a diferença entre os rendimentos e espera-se que o padrão de viagens das mulheres passe a se assemelhar aos dos homens. Além disso, pode-se esperar que haja um aumento no uso do carro pelas mulheres devido à pressão para que deem conta tanto do trabalho formal como do trabalho doméstico pois se trata de um modo que confere flexibilidade de horário e de trajeto, bem como relativa rapidez. Entretanto a presença de criança na família gera o seguinte efeito: a maternidade reduz a probabilidade de o uso do carro por mulheres e a paternidade a aumenta para homens. (Best e Lanzendorf, 2005).

%Labour force participation on the other hand, especially when part-time, intensifies car use for both genders. (BEST2005)
% While parenthood reduces the odds of car use by women, it increases men’s car use. (BEST2005)

% Women usually have shorter work-trips, use public transport more frequen- tly, and tend to trip-chain more often than men (Jones et al., 1983; Hjorthol, 2000; Rosenbloom, 1998; McGuckin and Murakami, 1999; Root and Schintler, 1999)

%``os relatos dos cronistas, viajantes e historiadores do período nos exibem um quadro em que a menina ou a mulher [burgues] candidata ao casamento é extremamente bem cuidada, é trancafiada nas casas, etc.''\cite{DINCAO2012}

% Ao analisar as viagens motivo trabalho há diferenças significativas indicadas por recentes pesquisas de países ocidentais. Neste tipo de viagem as mulheres costumam usar mais transporte público do que os homens e fazer mais cadeias de viagens, ou seja, usar uma combinação de diversos modos de transporte na consecução de uma viagem (Hjorthol, 2000).

% Entende-se que o principal motivo dessas diferenças seja a desigual distribuição do trabalho doméstico, que leva a diferentes padrões de atividades e, assim, de viagens. Na Alemanha, por exemplo, a quantidade de tempo dispendida pelos homens com tarefas da casa se manteve constante nas últimas décadas, não tendo alteração significativa com o crescimento da participação feminina no mercado de trabalho (Künzler, 1994). O trabalho doméstico gera viagens, que Best e Lanzendorf denominam como viagens de “manutenção” do lar, isto é, aquelas relacionadas ao trabalho físico de emocional que reprodução da força de trabalho ou da sociedade em si. Novamente nota-se a mulher ligada a atividades de reprodução, apartada da esfera produtiva – princípio da separação. São atividades representativas para compor essa demanda de viagens de “manutenção”: compras, cuidados com as crianças (por exemplo, ir e voltar do pediatra, levar e trazer da escola) e cuidados com idosos (por exemplo, acompanhar ao médico).

% Ter à disposição um carro para uso privado é o fator que mais influencia o seu uso em viagens motivo “manutenção”. Outros fatores que estimulam o uso do carro são: renda familiar e participação no mercado de trabalho. Se há mais renda familiar, há mais condições financeiras de adquirir um carro para tê-lo à disposição. Se há mais pessoas na família que fazem parte do mercado de trabalho, há mais interesse em comprar um carro, seja por aumento da renda familiar, seja pela necessidade de modo com flexibilidade de rota.

% As grandes compras geralmente são feitas utilizando-se o carro, especialmente num contexto de padrão de consumo em que se expandem os grandes hipermercados, no Brasil, que precisam de grandes áreas urbanas e vendem em grandes volumes. Desta forma, estes empreendimentos têm localização menos central e buscam estar próximos de grandes avenidas. Essas condições de contorno (facilidade de acesso por carro e estímulo a grande volume de compras) tendem a levar o usuário a preferir o carro como meio de transporte.

% Já as pequenas compras geralmente são feitas a pé e estas são majoritariamente feitas pelas mulheres. Trata-se da compra diária na padaria, na farmácia, no mercado do bairro, entre outras. Porém, no Brasil, as pesquisas Origem-Destino costumam ignorar as viagens feitas a pé num raio inferior a 500m, o que inclui grande parte das viagens de “manutenção” como as de pequenas compras de abastecimento doméstico (Metrô, 2007).

% A presença de criança é um fator que impacta bastante no padrão de atividades da família, de forma diferentes para pais e para mães. Um estudo feito na Alemanha indica que mães usam menos frequentemente o carro do que mulheres sem filhos, ao passo que pais usam mais o carro do que homens sem filhos (Best e Lanzerndorf, 2005). Esse comportamento pode se dever ao fato de que os homens, antes da paternidade, já estão mais familiarizados com o uso do carro e tendem a não trocar sua escolha modal. Vale ressaltar que as escolas, em especial que atendem as crianças nas primeiras idades, geralmente ficam próximas à residência e que a maior parte das viagens para levar crianças à escola são a pé, feita pelas mães, que se estiverem num raio inferior a 500m, no Brasil, passam despercebidas.

% Conflicting evidence exists in the literature on commuting about whether or not the greater household responsibilities of women lead to their widely observed shorter work trips compared to men. In light of changes in American household structures, this study reexamines the household responsibility hypothesis by focusing on household type (defined in terms of number of workers present in the home). Male and female work-trip distances are compared for Baltimore workers in single-worker households and for those in two-worker households. The findings support the household responsibility hypothesis by showing a larger and more significant sex disparity among respondents in two-worker households than among those in single-worker households even after controlling for other factors, including presence of children. These results, and the finding that married women have shorter work trips than married men, are in line with the general conclusions of some previous studies that the unequal division of labor within the household is partly responsible for the gender difference. \cite{IBIPO1992}

% No Brasil, o olhar integrador entre transportes, planejamento urbano, meio ambiente e aspectos sociais tem sido cada vez mais frequente. Dois exemplos são o Estatuto da Cidade (\citeyear{ESTATUTOCIDADE}), obrigatório para cidade com mais de 20 mil habitantes, e o Plano Nacional de Mobilidade Urbana (\citeyear{PNMU}), obrigatório para cidades com mais de 500 mil habitantes. Tratam-se de dois instrumentos legais que norteiam elaboração de políticas públicas e, de acordo com \citeauthoronline{IEMA2010}(\citeyear{IEMA2010}):

%\begin{citacao}
%Estatuto da Cidade estabelece o direito às cidades sustentáveis para a atual e as futuras gerações, [sendo esse direito] compreendido como o acesso ao solo urbano, moradia, saneamento, infraestrturua, trabalho, lazer e serviços públicos.
%\end{citacao}

%\begin{citacao}
%A descrição e análise dos fenômenos de megalopolização que ocorrem durante os últimos 500 anos surpreendem pela convergência de padrões na maioria das megalópoles latino-americanas aqui apresentadas. Eles não podem ser atribuídos à história, mas apontam para forças macroestruturais que promovem um desenvolvimento urbano que converge para a ``insustentabilidade'' das megalópoles na era da globalização. \cite{FREITAG2007}
%\end{citacao}
