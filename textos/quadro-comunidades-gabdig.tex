\begin{quadro}[htb]
    \IBGEtab{
        \renewcommand{\arraystretch}{1.4}
        \ABNTEXfontereduzida
    \caption{\label{qua:gabinetedigital-bate-bola-positivos}Gabinete Digital - Pontos positivos} %todo
  }{%
    %\resizebox{1.5\textwidth}{!}{
      \tiny
      \begin{tabular}{|p{2.8cm}|p{12cm}|}
        \hline
          \headerCenterCell{Nome} &
          \headerCenterCell{Aponte os 3 principais pontos positivos do portal}\\
        \hline \hline
          &
          Contribuição importante para uma necessária \textbf{mudança de cultura} na gestão pública do estado do Rio Grande do Sul e para uma \textbf{mudança de linguagem} dentro das estruturas formais de participação \\
        %\hline
          \textbf{Vinicius Wu} &
          Possibilidade efetiva de estabelecer ligações de novo tipo entre Estado e sociedade pela conformação de um espaço \textbf{público não estatal}, onde a cidadania perceba que o ambiente não é pertencente a um determinado governo \\
        %\hline
          &
          Aumento da \textbf{porosidade} do Estado - como exemplo, tem-se o Gabinete Digital sendo um canal de interlocução fundamental entre Estado e sociedade durantes os protestos de 2013 \\
        %\hline
          &
          Demonstração de que é possível que os governos adotem novas tecnologias de informação e comunicação para proporcionar o fortalecimento da cidadania \\
        \hline
          &
          Inovação \\
        %\hline
          \parbox{2.8cm}{\textbf{Luiz Carlos} \\\textbf{Damasceno Junior}} &
          Caráter público e não institucional \\
        %\hline
          &
          Capacidade de tensionar e trazer novas dinâmicas de participação p/ dentro do governo (questionar por dentro / caráter ombusdman) \\
        \hline
          &
          Inovação \\
        %\hline
          \parbox{2.8cm}{\textbf{Uirá Porã}  \\\textbf{Carmo Maia}} &
          Laboratório \\
        %\hline
          &
          Respaldo político do chefe do executivo \\
        \hline
          &
          Ferramenta que aproxima o cidadão para além das eleições \\
        %\hline
          \textbf{Guilherme Guerra} &
          Ferramenta de código aberto, disponível para todos que quiserem participar tecnicamente \\
        %\hline
          &
          Início de uma nova forma de pensar a política e o governo \\
        \hline
      \end{tabular}
      \normalsize
    %}
  }{%
    \fonte{compilação própria}
  }
\end{quadro}

\begin{quadro}[htb]
    \IBGEtab{
        \renewcommand{\arraystretch}{1.4}
        \ABNTEXfontereduzida
    \caption{\label{qua:gabinetedigital-bate-bola-negativo}Gabinete Digital - Pontos negativos}
  }{%
    %\resizebox{1.5\textwidth}{!}{
      \tiny
      \begin{tabular}{|p{2.8cm}|p{12cm}|}
        \hline
          \headerCenterCell{Nome} &
          \headerCenterCell{Aponte os 3 principais pontos negativos do portal}\\
        \hline \hline
          &
          Não se conseguiu introduzir alguns elementos indutores que inscrevessem na estrutura do Estado mecanismos que cristalizassem a mudança cultural inicialmente almejada \\
        %\hline
          \textbf{Vinicius Wu} &
          Não se conseguiu avançar a articulação da experiência do Gabinete Digital a outras estratégias que mobilizassem outros órgãos, no sentido de abertura de dados e do uso de tecnologias livres \\
        \hline
          &
          Falta de institucionalidade \\
        %\hline
          \parbox{2.8cm}{\textbf{Luiz Carlos} \\\textbf{Damasceno Junior}} &
          Certa dificuldade de escoar demandas que chegam \\
        %\hline
          &
          Não ter conseguido criar um espaço permanente de participação/relacionamento com o cidadão \\
        \hline
          &
          Falta de respaldo do resto do governo \\
        %\hline
          \parbox{2.8cm}{\textbf{Uirá Porã}  \\\textbf{Carmo Maia}} &
          Falta de interesse da população no processo de governança como um todo \\
        %\hline
          &
          Falta de institucionalidade dessas ferramentas digitais e das estruturas de participação – processo mais claros e automáticos \\
        \hline
          &
          Ter uma equipe técnica pequena para a demanda e potencial do portal \\
        %\hline
          \textbf{Guilherme Guerra} &
          Não ter muita gente participando ativamente como comunidade de desenvolvimento técnico \\
        %\hline
          &
          Governo é lento e muito atrasado ainda, e o Gabinete Digital depende do governo para poder funcionar \\
        \hline
      \end{tabular}
      \normalsize
    %}
  }{%
    \fonte{compilação própria}
  }
\end{quadro}

\begin{quadro}[htb]
    \IBGEtab{
        \renewcommand{\arraystretch}{1.4}
        \ABNTEXfontereduzida
    \caption{\label{qua:gabinetedigital-bate-bola-desafios}Gabinete Digital - Desafios}
  }{%
    %\resizebox{1.5\textwidth}{!}{
      \tiny
      \begin{tabular}{|p{2.8cm}|p{12cm}|}
        \hline
          \headerCenterCell{Nome} &
          \headerCenterCell{Quais seriam os 3 maiores desafios do Portal?} \\
        \hline \hline
          &
          Aumentar capacidade de comunicação \\
        %\hline
          \textbf{Vinicius Wu} &
          Ampliar a participação por meio do Gabinete Digital e apropriação do portal por parte da população \\
        %\hline
          &
          Inspirar outros órgãos da administração pública \\
        \hline
          &
          Conseguir fazer com que não haja só uma caixinha Gabinete Digital na internet $\rightarrow$ lógica digital permeie toda a máquina pública $\rightarrow$ governança digital \\
        %\hline
          \parbox{2.8cm}{\textbf{Luiz Carlos} \\\textbf{Damasceno Junior}} &
          Como o governo pode se repensar p/ oferecer ao cidadão explorar o acesso do cidadão à internet \\
        %\hline
          &
          Conseguir c/ q o participação participe desde o início da política pública (desde a priorização no início até o fim do processo, obter um feed-back) \\
        \hline
          &
          Abrir mais o governo, estabelecendo canais efetivos e permanentes para participação e intervenção da população, pois hoje o Gabinete Digital atua muito como intermediário entre sociedade de governo, logo, ainda não se configura um canal que abre o poder, mas um canal que intermedeia os processos de diálogo. \\
        %\hline
          &
          Se firmar e conseguir quebrar barreiras em relação à questão de linguagem, formato e postura não só com o governo, mas também com a população \\
        %\hline
          \parbox{2.8cm}{\textbf{Uirá Porã} \\\textbf{Carmo Maia}} &
          Ter relevância e respaldo como um canal a que de fato se pode recorrer, que gerará alguma consequência (resposta/resultado) \\
        %\hline
          &
          Conseguir massificar os acessos (Carmo aponta que há aproximadamente 10mil acessos mensais no site, uma marca baixa) \\
        %\hline
          &
          Grande participação no Facebook, mas dificuldade de converter isso para a plataforma - tem-se até o crescimento do acesso ao site, mas que não se traduz em engajamento participativo nas ações propostas. \\
        \hline
          &
          Ter uma equipe técnica que desse conta de todas coisas fantásticas que ele poderia fazer \\
        %\hline
          \textbf{Guilherme Guerra} &
          Vencer preconceito e a forma antiga de governo que temos \\
        %\hline
          &
          Superar os desafios políticos das engrenagens do sistema, de como as coisas funcionam \\
        \hline
      \end{tabular}
      \normalsize
    %}
  }{%
    \fonte{compilação própria}
  }
\end{quadro}

\begin{quadro}[htb]
    \IBGEtab{
        \renewcommand{\arraystretch}{1.4}
        \ABNTEXfontereduzida
        \caption{\label{qua:gabinetedigital-bate-bola-proximos-passos}Gabinete Digital - Próximos Passos}
  }{%
    %\resizebox{1.5\textwidth}{!}{
      \tiny
      \begin{tabular}{|p{2.8cm}|p{12cm}|}
        \hline
          \headerCenterCell{Nome} &
          \headerCenterCell{Quais são / deveriam ser os próximos passos e principais metas do Portal para os próximos 4 anos?} \\
        \hline \hline
          \textbf{Vinicius Wu} &
          Não houve tempo hábil na entrevista para perguntar \\
        \hline
          &
          \\
        %\hline
          \parbox{2.8cm}{\textbf{Luiz Carlos} \\\textbf{Damasceno Junior}} &
          Não respondeu esta pergunta \\
        %\hline
          &
          \\
        \hline
          &
          Estabelecer o login cidadão porque é estratégico - através dessa conta o governo pode conhecer melhor e estabelecer uma interação de relacionamento com cada cidadão, pois hoje se participa um conjunto de pessoas numa consulta e um conjunto de pessoas noutra consulta, não se sabe se são as mesmas pessoas ou não, nem o governo consegue estabelecer uma relação mais amigável com elas \\
        %\hline
          \parbox{2.8cm}{\textbf{Uirá Porã} \\\textbf{Carmo Maia}} &
          Adotar as ferramentas e tecnologias de participação desenvolvidas no Gabinete Digital e implementá-las em outras áreas e setores do governo gaúcho \\
        %\hline
          &
          Estruturar e abrir a gestão da informação do governo do estado do Rio Grande do Sul \\
        \hline
          &
          Fortalecer uma comunidade/equipe de desenvolvimento \\
        %\hline
          \textbf{Guilherme Guerra} &
          Fortalecer ele com essa forma de fazer política, de maneira inovadora \\
        %\hline
          &
          Ele espalhar, “que cada secretaria tenha seu gabinete digital, por exemplo, para agilizar os processos” \\
        \hline
      \end{tabular}
      \normalsize
    %}
  }{%
    \fonte{compilação própria}
  }
\end{quadro}


