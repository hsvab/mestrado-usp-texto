\clearpage
\section{Sustentabilidades}

Por definição de sustentabilidade encontram-se nos dicionários descrições bastante simples e amplas, que podem ser resumidas como ``a qualidade de ser sustentável'' \cite{MICHAELIS2014}, o que posterga a dúvida para a questão: o que é ser sustentável? Segundo \citeauthoronline{BLACK2010} (\citeyear{BLACK2010}), é aquilo que pode ser mantido ou que dure. É evidente que tal durabilidade não é eterna, mas por um determinado período. A ideia da permanência leva a crer que tal período seja longo, que relacione-se à perspectiva de longo prazo, mas de quão longo se trata, é uma indefinição, até hoje. Hoje, inclusive, é cada vez mais frequente o uso do termo sustentável como modificador ao invés de sustentabilidade como um conceito fechado em si. Aqui, exploraremos o desenvolvimento  sustentável e o transporte sustentável como as ``sustentabilidades'' de interesse.

As primeiras preocupações e dicotomizações entre desenvolvimento e meio ambiente remontam ao fim da década de 1960, com o Clube de Roma%
\footnote{O Clube de Roma fora fundado em 1968 por Aurelio Peccei e Alexander King e consistia num grupo de pessoas ilustres (empresários, líderes religiosos, políticos, entre outros) que se reuniam para discutir assuntos ligados à política, economia e, também, meio ambiente. Para saber mais: \url{http://www.clubofrome.org/} Acesso em 06 de novembro de 2014.}, que será o berço da obra \emph{The Limits to Growth}. Este livro, publicado em 1972, problematiza pela primeira vez a questão do crescimento exponencial \emph{versus} a finitude dos recursos disponíveis e também se propõe a simular e tentar prever as consequências da interação antrópica com sistemas não-antrópicos \cite{MEADOWS1972}. Nesse mesmo ano, ocorre a Conferência sobre o Ambiente Humano das Nações Unidas em Estocolmo.

Em 1983, as Nações Unidas (ONU) fundam a Comissão Mundial sobre Meio Ambiente e Desenvolvimento (WCED), composta por 19 delegados de 18 países, com a missão de produzirem um estudo sobre desenvolvimento em escala global, considerando aspectos como sustentabilidade e meio ambiente num perspectiva de longo prazo. Assim, a expressão ``desenvolvimento sustentável'' aparece pela primeira vez em \citeyear{WCED1987}, no relatório \emph{Our Common Future} da WCED, também conhecido como \emph{Brundtland Report}%
\footnote{Gro Harlem Brundtland era o Primeiro Ministro da Noruega e foi quem comandou a Comissão Mundial sobre Meio Ambiente e Desenvolvimento (WCED). Fonte: \url{http://www.un-documents.net/our-common-future.pdf} Acesso em 06 de novembro de 2014.}, onde é apresentado o clássico conceito:

\begin{citacao}
desenvolvimento sustentável é aquele que satisfaz as necessidades do presente sem comprometer a capacidade das gerações futuras satisfazerem as suas próprias necessidades. Este conceito contém em si outros dois conceitos-chave: o de ``necessidades'', em particular as necessidades essenciais dos pobres do mundo, às quais deve ser dada prioridade absoluta; e a ideia de limitações impostas pelo estágio tecnológico e de organização social sobre a capacidade do meio ambiente de satisfazer as necessidades presentes e futuras. 
\cite[p.41]{WCED1987}
\end{citacao}   

Essa definição consolidou-se na Conferência das Nações Unidas sobre Meio Ambiente e Desenvolvimento ocorrida em 1992 no Rio de Janeiro, também conhecida como ECO-92. Em quase vinte anos, o conceito popularizou-se, ganhou robustez e também ficaram mais nítidas suas limitações de implementação. Um relatório de balanço da ONU publicado em 2010 \cite{ONU2010} indica haver convergência conceitual de que o desenvolvimento sustentável está alicerçado sobre três pilares: desenvolvimento econômico, equidade social e proteção ambiental. O mesmo documento reconhece, porém, que apesar de visionário e integrador, o conceito tem se mostrado de difícil implementação pelos países e pouco tem sido abraçado em sua completude pelas políticas das diversas nações. 

Embora o desenvolvimento sustentável pretenda englobar os três pilares, o desenvolvimento é frequentemente sinônimo de desenvolvimento econômico e a sustentabilidade fica muitas vezes compartimentada à questão ambiental \cite{ONU2010}. Um dos caminhos que vem sendo construído para tentar superar essa dificuldade é tornar o conceito menos difuso e mais palpável por meio de indicadores \cite{CHAMBERS2000,BOULANGER2008,BARRETT2010,FORTES2012} e metas (de preferência quantitativas) a serem atingidas num determinado prazo \cite{ONU2010,ONU2014}.

Outra saída, não excludente como esta recém apresentada, é adicionar outros pilares na conceituação do que seja desenvolvimento sustentável, como faz 
\citeauthoronline{BANISTER2005} (\citeyear{BANISTER2005}). Ele elenca outros dois fatores como fundamentais: (i) participação e (ii) governança. A dimensão da participação refere-se a envolver todas as pessoas interessadas e envolvidas no processo, a saber, indivíduos, empresas, indústrias e governos. Argumenta que criar excluídos do processo torna muito mais difícil desenvolver as estratégias necessária de mudança. A dimensão da governança incide diretamente no processo de tomada de decisão, logo, significa mudanças nas estruturas organizacionais para que sejam facilitadas decisões intersetoriais.

\citeauthoronline{BANISTER2005} (\citeyear{BANISTER2005}) destaca cinco lições aprendidas desde o \emph{Brundtland Report} a partir das experiências de sucessos e fracassos: (i) as medidas que implicam redução de consumo  ou impactam estilo de vida devem começar modestamente; (ii) desestímulo às emissões de dióxido de carbono (CO$_2$) deve contar com mecanismos fiscais; (iii) deve haver incentivos fortes em pesquisa e desenvolvimento em ciência e tecnologia na temática das mudanças climáticas; (iv) embora todos países devam contribuir para a diminuição de emissões de carbono, a liderança cabe às nações mais ricas; (v) é preciso ação imediata e incerteza não é uma boa razão para inação ou atitudes fracas.
Ao se falar em sustentabilidades, fala-se necessariamente de mudanças de paradigma, profundas, e que podem até mesmo ser inatingíveis \cite{GLASBY2002}, embora possam ser perseguidas. \citeauthoronline{ROGERS2000} (\citeyear{ROGERS2000}) aponta que essa mudança deve necessariamente compreender as ``relações entre cidadãos, serviços, políticas de transporte e geração de energia, bem como seu impacto total no meio ambiente local e numa esfera geográfica mais ampla'' e aponta ainda as cidades, pensadas como organismos vivos, cujo metabolismo é bastante linear, precisando tornar-se mais circular (ver figura \ref{fig:cidade-metabolismos}).

\begin{figure}[htb]%
    \caption{\label{fig:cidade-metabolismos}Cidades de metabolismo linear e circular}%
    \begin{center}%
        \includegraphics[width=0.80\textwidth]{./imagens/richard-linear-circular.jpg}%
    \end{center}%
    \fonte{\cite[p.31]{ROGERS2000}}
\end{figure}%

Do ponto de vista econômico, a obra \emph{Limits to Growth} fez escola e trouxe à baila a hipótese de que o crescimento econômico teria um teto, função dos recursos (naturais) disponíveis. Contudo, há estudiosos que se opõem a isso, como \citeauthoronline{KRUGMAN2014} (\citeyear{KRUGMAN2014}) em seu recente artigo \emph{Slow Steaming and the Supposed Limits to Growth}. O economista argumenta que é possível manter o crescimento econômico real (do PIB) e ainda assim reduzir a emissão de gases do efeito estufa. Para chegar a essa conclusão ele apresenta uma demonstração - bastante simplista - que considera o consumo de energia dos navios em função de suas velocidades e conclui ser possível manter um caudal econômico constante e, concomitantemente, diminuir o consumo de energia do sistema.

Entre as principais preocupações no âmbito ambiental estão a diminuição das reservas de petróleo, aquecimento global por conta da emissão de gases do efeito estufa, poluição (atmosférica, sonora e hídrica) e presença de chuva ácida. Diversos estudos indicam que existe correlação entre a ocorrência de diversos tipos de doenças (cardiorrespiratórias, câncer, entre outras) e a exposição a alguns poluentes presentes na atmosfera \cite{WHO2000,WHO2006,BRUNEKREEF2012,MIRANDA2012}. Em São Paulo, \citeauthoronline{GOUVEIA2006} (\citeyear{GOUVEIA2006}) observaram associação estatisticamente significante entre o aumento no nível de poluentes na atmosfera e o aumento de hospitalizações por causas diversas, em todos grupos etários estudados. Entre 1971 e 2001, as emissões de CO$_2$, indicado como o principal gás responsável pelo efeito estufa, aumentaram cerca de 60\% e a parcela cuja origem são os sistemas de transporte também aumentou de 19,3\% para 28,9\% \cite{BANISTER2005}.

Sob o prisma da equidade social, o acesso equânime a oportunidades de educação, trabalho, saúde e lazer é um dos pontos centrais. A equidade, associada à ideia do ``ser justo'', inevitavelmente referir-se-á à distribuição social de custos e benefícios, bem como em que grau essa distribuição é considerada adequada e que corrobore para a promoção da justiça \cite{LITMAN2006}. Aqui também o transporte tem papel estruturador já que pode ser o elemento que provê ou barra o acesso às oportunidades. \citeauthoronline{SANCHEZ2003} (\citeyear{SANCHEZ2003}) já apontavam que equidade seria um dos temas estratégicos nas políticas de transportes. As megalópoles latino-americanas são, por vezes, cidades ``partidas'' \cite{VENTURA2001} entre a ``legal'' e a ``real'' \cite{ALVA1997}%
\footnote{Estima-se que cerca de 40\% ou mais da população possui moradia em condição irregular \cite{FREITAG2007}.},
onde as vias de circulação frequentemente são cicatrizes no tecido urbano - por exemplo, em São Paulo, o ``Minhocão'' \cite{ABASCAL2010}, os monotrilhos \cite{ROLNIK2010} ou mesmo uma rua na favela de Paraisópolis (ver Figura \ref{fig:paraisopolis}), em São Paulo.

\begin{figure}[htb]%
    \caption{\label{fig:paraisopolis}Favela de Paraisópolis: sua parca arborização e a divisa com parte nobre do bairro Morumbi em São Paulo}%
    \begin{center}%
        \includegraphics[width=1.0\textwidth]{./imagens/paraisopolis.jpg}%
    \end{center}%
    \fonte{Foto da esquerda de Gustavo Roth; foto da direita de Tuca Vieira/Folha Imagens, disponível em: \url{http://www.scielo.br/scielo.php?script=sci_arttext&pid=S0103-49792010000200005} Acesso em 11 de novembro de 2014}
\end{figure}%

Pode-se observar nos três pilares clássicos do desenvolvimento sustentável o papel relevante dos transportes. \citeauthoronline{VASCONCELLOS2012} (\citeyear{VASCONCELLOS2012}) aponta ainda que a energia gasta na mobilidade por habitantes de uma cidade, ou seja, quanta energia os moradores de um município precisam para se deslocarem permite ter uma ideia do ``grau de sustentabilidade'' da mesma. Dessa maneira, cabe uma breve discussão sobre transporte sustentável.

Para Black, um transporte sustentável seria aquele que atende às ``atuais necessidades de transporte e mobilidade e não deve comprometer a capacidade das futuras gerações satisfazerem as suas próprias necessidades'' \cite[p.151]{BLACK1996}, e também que ``provê transporte e mobilidade com combustíveis renováveis, minimizando as emissões prejudiciais ao ambiente local e globalmente, e prevenindo fatalidades, lesões e congestionamentos desnecessários'' \cite[p.12]{BLACK2010}.

Banister (\citeyear{BANISTER2005,BANISTER2008}) aponta algumas medidas a serem perseguidas para que se possa alcançar um transporte sustentável:
\begin{compactitem}[]
\item (i) reduzir a necessidade de viajar;
\item (ii) encorajar a troca para modos de transporte coletivo ou não motorizado;
\item (iii) reduzir o comprimento das viagens;
\item (iv) incentivar a adoção de sistemas e tecnologias de transporte mais eficientes, tanto para carga quanto para passageiros;
\item (v) reduzir a utilização de carros e caminhões de carga nas áreas urbanas;
\item (vi) reduzir, na fonte, ruídos e emissões dos veículos,
\item (vii) incentivar a utilização mais eficiente e ambientalmente consciente do estoque de veículos;
\item (viii) melhorar a segurança de pedestres e de todos usuários das (rodo)vias;
\item (ix) melhorar a atratividade das cidades para seus moradores, trabalhadores, compradores e visitantes.
\end{compactitem}

Embora cientes (organismos internacionais, governos e comunidades científicas) de medidas que corroborariam para o estabelecimento de um transporte sustentável, os padrões de mobilidade observados indicam uma dependência cada vez maior do automóvel (com poucas exceções), seja nos países desenvolvidos \cite{BANISTER2005}, seja nos países em desenvolvimento \cite{VASCONCELLOS2012}. \citeauthoronline{BANISTER2005} (\citeyear{BANISTER2005}) informa que entre 1984 e 1994 houve um aumento de 31\% na posse de veículos e que estimava-se chegar a 50\% em 2020. Ele também indica que a maior parte da frota (70\% em 2005) encontrava-se nos países desenvolvidos.

Entretanto, isso não significa que a posse de carros não esteja crescendo nos países em desenvolvimento. No Brasil, a frota de automóveis vem crescendo desde 1960 conforme pode ser observado na Tabela \ref{tab:venda-veic-br}, sendo que em 2009, cerca de 79\% da frota total de veículos era composta por automóveis \cite{VASCONCELLOS2012}. Esse fato somado ao de que o automóvel é o modo que apresenta o maior consumo energético (ver Tabela \ref{tab:gep-modo}) levam a concluir que o desenvolvimento no Brasil também trilha o caminho da insustentabilidade.

\clearpage
\begin{table}[htb]
    \IBGEtab{%\renewcommand{\arraystretch}{1.5}%%\ABNTEXfontereduzida%
	    \renewcommand{\arraystretch}{1.5}
        \caption{Venda interna de veículos no Brasil entre 1960 e 2009}
		\label{tab:venda-veic-br}
    }{%
	    \begin{tabular}{p{2.00cm} P{4.0cm} P{4.0cm} P{4.0cm}}
            \toprule
	           \headerTabCenterCell{Ano} &
		       \headerCell{Autos} &
		       \headerCell{Total} &
		       \headerCell{Fator de crescimento (total)} \\
		    \midrule \midrule
		        1960&
		        40.980&
		        131.499&
		        1\\
		    \midrule
		        1970&
		        308.024&
		        416.704&
		        3,2\\
		    \midrule
		        1980&
		        739.028&
		        980.261&
		        7,5\\
		    \midrule
		        1990&
		        532.906&
		        712.741&
		        5,4\\
		    \midrule
		        2000&
		        1.176.774&
		        1.489.481&
		        11,3\\
		    \midrule
		        2009&
		        2.474.764&
		        3.141.240&
		        23,9\\
		    \bottomrule
		\end{tabular}
    }{%
		\fonte{Adaptado de \cite[p.29]{VASCONCELLOS2012}}
		}
\end{table}

\begin{table}[htb]
    \IBGEtab{%\renewcommand{\arraystretch}{1.5}%%\ABNTEXfontereduzida%
	    \renewcommand{\arraystretch}{1.5}
        \caption{Consumo energético teórico dos modos de transporte em lotação plena}
		\label{tab:gep-modo}
    }{%
	    \begin{tabular}{p{4.00cm} P{4.0cm}}
            \toprule
	           \headerTabCenterCell{Modo de Transporte} &
		       \headerCell{gramas equivalentes de petróleo para mover um passageiro por um quilômetro}\\
		    \midrule \midrule
		        ônibus comum&
		        4,1\\
		    \midrule
		        metrô&
		        4,3\\
		    \midrule
		        motocicleta&
		        11,0\\
		    \midrule
		        automóvel&
		        19,3\\
		    \bottomrule
		\end{tabular}
    }{%
		\fonte{Adaptado de \cite[p.84]{VASCONCELLOS2012}}
		}
\end{table}

Se parece ilógica e insustentável a adoção do automóvel particular como modo principal de locomoção, por que ele continua tão bem cotado? A resposta a essa pergunta parece ser uma soma de fatores que o leva a ser um ícone, culturalmente simbólico e economicamente valorizado. 
Sobre o caráter simbólico, \citeauthoronline{BANISTER2005} (\citeyear{BANISTER2005},. p.05) expõe que o carro é visto como ``seguro, sempre disponível e nunca muito longe do seu motorista''. \citeauthoronline{URRY2001} (\citeyear{URRY2001}) indica ainda outros fatores que contribuem para esse \emph{status} do carro: 
\begin{compactitem}[]
\item (i) como um objeto manufaturado, nascido com o fordismo, é um ícone do sucesso capitalista; 
\item (ii) depois da moradia, é o principal bem de consumo que confere \emph{status} social ao indivíduo;
\item (iii) é um objeto de suficiente complexidade que sintetiza e ilustra um avanço tecnológico;
\item (iv) confere mobilidade individual e, portanto, liberdade para algumas escolhas como horários de saída e rotas adotadas;
\item (v) é revestido de um discurso pela mídia e pela indústria cultural que o liga ao sucesso e ao progresso.
\end{compactitem}

Sob a perspectiva econômica, na indústria brasileira, o ramo automobilístico tem tido um papel bastante central. Nos anos 1950 foram instaladas as três maiores montadoras à época na região de São Bernardo do Campo (SP), o que gerou emprego, aqueceu a indústria e também estimulou o nascimento de toda uma geração de motoristas de carro. Desde então, há uma pressão crescente por mais vias, maiores, melhores e mais fluidas.
%Em 1990, com a estabilização da economia e o controle da inflação, houve o fortalecimento do setor da construção civil e a popularização do financiamento de motos e carros. Esses elementos geraram uma megalópole com, por exemplo, \emph{shopping centers} que dispõem de gigantesca quantidade de vagas de estacionamento e condomínios com pelo menos uma vaga de garagem por apartamento, sem que tudo isso seja devidamente comportado pelos espaços de circulação.
Dado que o espaço é finito, ao aumentar os espaços de circulação, diminuem-se os espaços disponíveis para abrigarem as atividades das pessoas. Há mais de 50 anos \apudonline[p.63]{OWEN1956}{BLACK2010} já reconhecia que:

\begin{citacao}
O problema do congestionamento se tornou tão grande que muitas comunidades estão chegando à conclusão de que nunca haverá avenidas nem vagas de estacionamento suficientes que permitam o movimento de todas as pessoas em carros particulares.
\end{citacao}

A RMSP sofre das contradições de políticas que apontam para direções diferentes, quando não antagônicas. No âmbito do município de São Paulo, conta-se com a ``Lei de Mudanças Climáticas'' \cite{LEICLIMASP2009} que prevê a redução de 30\% nas emissões dos gases do efeito estufa, além de substituição integral do uso de combustíveis fósseis por renováveis na frota de transporte público. No âmbito estadual, o Plano de Controle de Poluição Veicular 2011-2013 \cite{PCPV2011} indica, entre outros objetivos, a adoção da inspeção ambiental de veículos, uma (única) medida que incide sobre o transporte privado individual. Já no âmbito federal, para garantir aquecimento econômico e minimizar a taxa de desemprego, o Imposto sobre Produtos Industrializados (IPI) dos carros nacionais novos 1.0 foi a zero no primeiro semestre de 2012, sendo que até o final de 2014 não terá retornado ao patamar dos 11\% \cite{FAZENDA2014}.

Em alguma medida o conjunto das políticas públicas transparece um desejo de não restringir a posse do carro, mas seu uso. Ou seja, deseja-se ao mesmo tempo desviar do impacto econômico que uma diminuição das vendas de carros geraria e regular o uso dos automóveis. Essa é a abordagem liberal que diversas cidades, de vários países do mundo vem adotando. Isto é, não se deseja impor restrições legais ou econômicas, mas entender e estimular comportamentos mais interessantes para o conjunto da sociedade e que contribua para a construção de cidades mais sustentáveis. Todavia, \apudonline[p.7]{GILBERT2000}{BANISTER2005} deixa o alerta de que ``há uma ligação entre a posse do carro e uso do carro, e qualquer estratégia coerente para reduzir o uso do carro está fadada ao fracasso se realmente não abordar a causa da mobilidade insustentável, ou seja, o carro''.
